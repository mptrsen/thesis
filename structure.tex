\documentclass[a4paper]{scrartcl}
\usepackage[T1]{fontenc}			% International font encoding
\usepackage[utf8]{inputenc}		% input encoding: utf-8

%--------------------------------------------------
% \usepackage[german]{babel}		% German language 
\usepackage{lmodern}					% font looks better on screen
\usepackage{microtype}				% even better
\usepackage[cm]{fullpage}			% more space on the page
% \usepackage{natbib}
% \bibpunct{[}{]}{,}{n}{}{;}		% citing format:
% 1. opening bracket symbol, default = (
% 2. closing bracket symbol, default = )
% 3. punctuation between multiple citations, default = ;
% 4. 'n' for numerical style, 's' for numerical superscript style, default = author-year;
% 5. punctuation between the author names and the year
% 6. punctuation between years or numbers when common author lists are suppressed (default = ,);

% Math symbols and stuff
%--------------------------------------------------
% \usepackage{amsmath}					% all the fancy math stuff
% \usepackage{amssymb}					% more math symbols
% \usepackage{wasysym}					% even MORE math symbols
% \usepackage{upgreek}					% upright Greek letters

% Enhanced formatting
%--------------------------------------------------
% \usepackage{color}						% enable color
% \usepackage{colortbl}					% colored tables
% \usepackage{graphicx}					% fancy graphics
% \usepackage{wrapfig}					% wrap figures
% \usepackage{tabularx}					% better tables
% \usepackage{enumerate}				% customize numbered lists
% \usepackage{tocloft}					% fancy table of content
% \usepackage{caption}					% extended captioning, e.g. without numbers
% \usepackage{fancyhdr}					% fancy header customisation
% \usepackage[parfill]{parskip} % vspace instead of indentation
% \usepackage{geometry}					% customize page size, margins etc

% Extra stuff, glossary, index etc
%--------------------------------------------------
% \usepackage{glossaries}				% glossaries
% \usepackage{makeidx}					% index
% \usepackage{showidx}					% print index items for debugging
% \usepackage{url}
% \usepackage{hyperref}					% should be loaded last
% \usepackage{syntonly}					% check for syntax only; much faster
% \syntaxonly										% comment if output is desired
% \renewcommand{\cftsecleader}{\cftdotfill{\cftdotsep}}	% dots in TOC

% Stuff that doesn't fit in any category
% \usepackage{lastpage}					% provides total number of pages, use with:
%																% \pageref{LastPage}
%-------------------------------------------------- 

\begin{document}

\section{Introduction}

\subsection{Summarize thesis}

\subsection{Orthology}

\begin{itemize}
	\item Where does the term come from?
	\item Similarity vs. homology
	\item Orthology
	\item Paralogy:
	\begin{itemize}
		\item Inparalogy
		\item Outparalogy
	\end{itemize}
	\item Xenology
	\item Why is orthology important?
	\begin{itemize}
		\item Phylogenetics
		\item Other uses (protein families, functional annotation, \ldots)
	\end{itemize}
\end{itemize}

\subsection{How can orthology be assessed?}

\begin{itemize}
	\item Tree-based strategies
	\begin{itemize}
		\item Pros
		\item Cons
		\item Implementation examples
	\end{itemize}
	\item Graph-based strategies
	\begin{itemize}
		\item Pros
		\item Cons
		\item Implementation examples
	\end{itemize}
	\item Reciprocal best hits, triangulation, reference proteomes
\end{itemize}

\subsection{Example: The HaMStR approach}

\begin{itemize}
	\item How HaMStR does it
	\item Shortcomings
\end{itemize}

\subsubsection{Hidden Markov Models}

\begin{itemize}
	\item Theory
	\item Sequence weighting
\end{itemize}

\subsubsection{Room for improvement on the HaMStR approach}

\begin{itemize}
	\item Frameshift-corrected, corresponding nucleotide output
	\item Performance issues
	\item Bugs
	\item Modularity (enabling reciprocal HMM search)
\end{itemize}

\clearpage

\section{Methodology}

\subsection{Graph-based approach}

\begin{itemize}
	\item Explain the graph
	\item Reciprocal best hit, reference proteomes (triangulation)
\end{itemize}

\subsection{Algorithm}

\begin{itemize}
	\item Analysis
	\item Reporting
	\item Extensibility
	\item Future development
\end{itemize}

\subsection{Programs}

Description of these programs and why they are used:

\begin{itemize}
	\item MySQL
	\item HMMer3
	\item BLAST+
	\item Exonerate
\end{itemize}

\subsection{Config file-driven}

Advantages of using a config file vs. command-line options

\subsection{Database design}

\begin{itemize}
	\item Indexes
	\item Unique indexes
	\item InnoDB vs. MyISAM
	\item Example queries
\end{itemize}

\subsection{Performance tweaks}

\begin{itemize}
	\item Using native Perl functions instead of shelling out
	\item One BLAST database comprising all ``core'' taxa proteomes
	\item RDBMS vs. flat files
	\item LOAD DATA 
	\item Disabling MyISAM indexes during upload
	\item Transactions
\end{itemize}

\subsection{SHA-256 checksums as sequence identifiers}

\begin{itemize}
	\item Reasoning
	\item Collision probability
\end{itemize}

\clearpage

\section{Results}

\subsection{Benchmarks}

Run a couple of transcriptomes from 1KITE, compare with HaMStR:

\begin{itemize}
	\item Hit ratio
	\item Outliers
\end{itemize}

\subsection{HMM sensitivity}

Report of the conducted tests in spring 2012: Phylogenetic diversity in a HMM only marginally affects its specificity.

\section{Discussion and outlook}

\begin{itemize}
	\item Summary
	\item Future developments
\end{itemize}

\end{document}
