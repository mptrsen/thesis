Strategies for orthology prediction:

Tree reconciliation

\begin{itemize}
	\item topology of a gene tree compared with that of the chosen species tree
	\item reconciled by maximum parsimony $\rightarrow$ reflects ortholog
		relationships
	\item genome-wide application precluded by:
	\begin{itemize}
		\item horizontal gene transfer, especially in prokaryotes (widespread HGT
			invalidates the very notion of a species tree)
		\item computationally expensive
	\end{itemize}
\end{itemize}

\begin{description}
	\item[\cite{mirkin1995}] Tree-based approach to orthology prediction
	\item[\cite{page1998}] Tree reconciliation method
	\item[\cite{yuan1998}] Tree-based approach to orthology prediction
	\item[\cite{kuzniar2008}] Review of approaches
\end{description}

most studies employ simplifications/shortcuts $\rightarrow$ graph-based approaches:

Triangulation

\begin{itemize}
	\item OrthoDB 
	\item COG (KOG, EGO, etc)
\end{itemize}

Reciprocal best hit (RBH) 

\begin{itemize}
	\item RBH strategies recover only one-to-one orthologs (the bidirectional best
		hit)
	\item InParanoid (BLAST based)
\end{itemize}

Markov clustering:

\begin{itemize}
	\item OrthoMCL
\end{itemize}

Homoplasy: similarity in unrelated organisms. Homoplasies demonstrate adaptation
in the living world.

Homoplasy \citep{lankester1870}: It may be said that the term ``analogy'',
already in use, is sufficient to indicate what is here termed ``homoplasy''; but
analogy has had a wider signification given to it, in which it is found very
useful to employ it, and if could not be used with any accuracy in place of
homoplasy.  \emph{Any} two organs having the same function are analogous,
whether closely resembling each other in their structure and relation to other
parts or not, and it is well to retain the word in that wide sense. Homoplasy
includes all cases of close resemblance of form which are not traceable to
homogeny, all \emph{details} of agreement not homogenous, in structures which
are broadly homogenous, as well as in structures having no genetic affinity.
