Strategies for orthology prediction:

Tree reconciliation

\begin{itemize}
	\item topology of a gene tree compared with that of the chosen species tree
	\item reconciled by maximum parsimony $\rightarrow$ reflects ortholog
		relationships
	\item genome-wide application precluded by:
	\begin{itemize}
		\item horizontal gene transfer, especially in prokaryotes (widespread HGT
			invalidates the very notion of a species tree)
		\item computationally expensive
	\end{itemize}
\end{itemize}

\begin{description}
	\item[\cite{mirkin1995}] Tree-based approach to orthology prediction
	\item[\cite{yuan1998}] Tree-based approach to orthology prediction
	\item[\cite{kuzniar2008}] Review of approaches
\end{description}

most studies employ simplifications/shortcuts $\rightarrow$ graph-based approaches:

Triangulation

\begin{itemize}
	\item OrthoDB 
	\item COG (KOG, EGO, etc)
\end{itemize}

Reciprocal best hit (RBH) 

\begin{itemize}
	\item InParanoid (BLAST based)
\end{itemize}

Markov clustering:

\begin{itemize}
	\item OrthoMCL
\end{itemize}

