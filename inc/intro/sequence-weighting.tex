\label{sec:hmmtest}
To avoid statistical bias by uneven phylogenetic representation when creating a
HMM, HMMER3 \citep{eddy2009} uses a position-based weighting scheme as applied
by \citet{henikoff1994} by default: 

\begin{quote}
	``A simple method to represent the diversity at a position is to award each
	different residue an equal share of the weight, and then to divide that
	weight equally among the [residue] sequences sharing the same residue. So, if
	in a position of a multiple alignment, $r$ different residues are represented,
	a residue represented in only one sequence contributes a score of $1/r$ to
	that sequence, whereas a residue represented in $s$ sequences contributes a
	score $1/rs$ to each of the $s$ sequences. For each sequence, the
	contributions from each position are summed to give a sequence weight.''
	\hfill\citep{henikoff1994}
\end{quote} 

Thus, if two or more sequences are very similar in a particular region, the
Henikoff weighting scheme penalizes that by weighting the sequences down
position-wise. Highly diverse positions, on the other hand, would receive a
bonus. 

In comparison to, \eg, two species $a$ and $b$ from the same taxonomic family, the
remotely related species $a$ and $c$ are expected to have more divergent
sequences. Highly conserved domains may still be similar, but for the most part,
more divergence is expected. Under the Henikoff weighting scheme, closely
related domains will receive a penalty, while divergent ones are upweighted.
Because $a$ and $b$ are expected to have more positions in common, which results
in downweighting of a larger percentage of their entire sequences, these two
sequences are each ``worth'' less than the more remotely related sequence $c$.
Two identical sequences would each receive half the weight that one sequence
would.
