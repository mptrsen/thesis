With next-generation, high-throughput technologies providing vast amounts of
sequence data, it becomes more important to classify conserved genes according
to their homologous relationships. It is worth mentioning here that homology is
a concept of quality, not quantity (\cite{reeck1987}) and thus indivisible. Two
sequences can be \emph{similar} by a percentage (e.g., amino acid positional
identity), but they are either homologous or they are not. The same follows for
orthologs, paralogs, and xenologs. This distinction is important because
homology implies a genealogical relationship, whereas similarity does not.

A simple alignment or scoring (i.e., similarity) search cannot separate
homologous from merely similar sequences. 

See figure \ref{fig:hamstr}.
