Phylogenetic analysis searches for evidence of shared ancestry and thus, when
based on the analysis of genes, requires orthology among the compared genes. To
make use of the speciation information contained in the ortholog property, it is
important to classify genes according to their homologous relationships.
Homology is a concept of quality, not quantity \citep{reeck1987} and thus
indivisible. Two genes can be \emph{similar} by a percentage (e.g., amino acid
positional identity), but they are either homologous or they are not. The same
follows for orthologs, paralogs, and xenologs. The distinction between
homologous and not homologous is important because homology implies a
genealogical relationship, whereas similarity does not. Similarity can also be
the result of other evolutionary processes, such as convergence, which results
in \emph{analogy}. All variants of similarity among unrelated organisms can be
grouped under the term \emph{homoplasy}\footnote{\emph{homos} (``equal''),
\emph{plasis} (``shaping'')}, which was introduced by \citet{lankester1870} and
encompasses similarity but excludes homology and its subforms \citep{page1998}.

A simple gene alignment or scoring (i.e., a measure of similarity) alone cannot
separate homologous from non-homologous, but similar sequences
\citep{eisen1998}. Since orthology and paralogy are notions that describe
evolutionary events, they can be directly coupled to the topology of a species
tree. This is how \emph{tree reconciliation} strategies work, in contrast to
\emph{graph-based} strategies, which are explained later: the topology of a gene
tree is compared to the topology of the species tree that contains the chosen
analyzed genes. The two trees are reconciled using an optimality criterion such
as maximum parsimony, i.e., the minimal possible number of evolutionary events
(speciation, gene duplication, gene loss). In order to infer a reconciled tree,
two major approaches are frequently applied: \emph{consensus} tree inference and
\emph{approximation} (\autoref{fig:consens-approx}). Both algorithms consider a
phylogenetic tree a formal mathematical object.

In the consensus approach, a tree is searched for that combines the features
common to both the gene and species tree. The consensus tree is rarely fully
resolved since the consensus criterion leads to clustering
(\autoref{fig:consens-approx}c). Increasing the number of initial trees, e.g.,
gene trees, can reduce the number of common clusters, and the concept of a
``common cluster'' can be altered, e.g., to include only those clusters that
occur in 80\% or more of the initial trees, but in most cases, clustering cannot
be avoided altogether \citep{mirkin1995}.

\begin{figure}[t]
	\centering
	\def\svgwidth{0.8\textwidth}
	\input{img/consens-approx.pdf_tex}
	\caption[Consensus and approximation trees]{
		Consensus and approximation results of reconciling trees a) and b):
		consensus tree c) and approximation trees d) and e). Note the clustering in
		the consensus tree as a result of joining common features of the original
		trees, and the different topology in the approximation trees. Graphic
		based on \cite{mirkin1995}.
	}
	\label{fig:consens-approx}
\end{figure}


In the approximation approach to tree reconciliation, a similarity measure is
used across the initial trees. The algorithm tries to find a topology that
maximizes the total similarity (minimizes the total dissimilarity) to the
initial trees. However, these measures have no explicit biological meaning
\citep{mirkin1995}, and multiple approximation tree topologies can result for a
given set of starting trees, leading to ambiguous interpretations. 

Tree reconciliation strategies are subject to a number of restrictions when
applied on a genomic scale. The widespread presence of horizontal gene transfer,
which is very common in prokaryotes, invalidates the very notion of a species
tree \citep{doolittle2000}, because the manifold xenologous relationships
between prokaryote genes would show up in a reconciled tree in a partly or
completely unresolved topology.  In addition, reconstructing a phylogenetic tree
requires a multiple sequence alignment of the analyzed genes, which can be
automated using algorithms such as MAFFT \citep{katoh2005}, MUSCLE
\citep{edgar2004}, CLUSTALO \cite{sievers2011}, or others, but not to perfection
\citep{thompson2011}.  Further, the reconstruction is computationally expensive,
and doing so for all genes in sequenced genomes is a major challenge for modern
software engineering and computer systems. Nevertheless, a number of
implementations have developed; among the most recent are Orthostrapper
\citep{jothi2006} and LOFT \citep{van_der_heijden2007} as well as \mbox{COCO-CL}
\citep{storm2002}, which does not use a tree reconciliation algorithm, but
builds a hierarchy of sets of outparalogs. 

Confronted with these fundamental problems and practical obstacles of tree-based
orthology assessment strategies, most genome-wide studies employ a different
approach. Assuming that the nucleotide or amino acid sequences of orthologous
genes or proteins are more similar to each other than they are to other genes or
proteins in the compared genomes, a \emph{graph-based} strategy can be used. All
graph-based strategies use some form of symmetrical best hit criterion to
distinguish orthologs from paralogs or other homolog subtypes. The validity of
this criterion is made plausible by the above-mentioned ortholog conjecture; the
opposite (a gene from one genome being more similar not to its ortholog but to a
paralog from another genome) would require a substantially different
evolutionary rate in paralogs to create a large enough similarity offset
compared to orthologs \citep{koonin2005}.

In a graph-based approach, an algorithm creates a two-dimensional graph with
genes (or proteins) as nodes and evolutionary relationships as edges. The
relationships are inferred using a search algorithm such as Needleman-Wunsch
\citeyearpar{needleman1970}, Smith-Waterman \citeyearpar{smith1981}, or BLAST
\citep{altschul1990}, and clusters of orthologous genes are created
(\autoref{fig:graph-based-strat}b). In the event of a gene loss of, e.g., the
blue human gene shown in \autoref{fig:graph-based-strat}a, however, a simple
best-hit search algorithm would incorrectly infer an orthologous relationships
between the blue dog and the red human gene. In order to avoid these mistakes,
an additional criterion for orthology detection is introduced: the genes in
question must be \emph{bidirectional best hits} (BBH), also called reciprocal
best hits (RBH).  Since orthology is also a symmetric relation, this makes sense
\todo{rephrase}.

\begin{figure}[t]
	\centering
	\def\svgwidth{0.8\textwidth}
	\input{img/graph-based-strat.pdf_tex}
	\caption[Graph-based strategy]{Graph-based approach to asses gene orthology. 
		\textbf{a)} In the evolutionary scenario of a gene family with two speciation
			events S\textsubscript{1} and S\textsubscript{2} and one intraspecific
			gene duplication event D\textsubscript{1}, the gene in the frog is
			orthologous to all other genes. The red and blue genes are orthologs among
			themselves, but paralogs to each other.
		\textbf{b)} The corresponding graph. Genes are represented as circles and
			evolutionary relationships by edges. While the red and blue genes form
			one-to-one relationships among themselves, the gene in the frog has a
			one-to-many relationship to the red and blue genes. A bidirectional best
			hit search strategy would detect only the highest scoring one of these
			pairs. 
		Graphic from \citet{altenhoff2012-1}.
	}
	\label{fig:graph-based-strat}
\end{figure}


Graph-based strategies are computationally much less expensive than tree-based
strategies, since only two pairs of sequences are compared with and thus aligned
to each other for each gene: the time complexity scales quadratically with the
total number of genes \citep{altenhoff2012-1}. The specificity and sensitivity
of graph-based methods depend on the search algorithm and the scoring scheme
\citep{hulsen2006}.

Many implementations of graph-based algorithms have been introduced and improved
on the basic BBH strategy. Among the most widely known are COG (clusters of
ortholog groups) \citep{tatusov2003}, Inparanoid \citep{ostlund2010}, and
OrthoMCL \citep{li2003} as well as OrthoDB \citep{waterhouse2011}. These and
many more are reviewed by, e.g., \citet{kuzniar2008} and \citet{forslund2011}.
COG groups together genes that are BBH using BLAST. Inparanoid uses BLAST BBH as
well, but assesses only pairwise orthology between species. OrthoMCL builds on
the Inparanoid approach, but extends it using a Markov cluster algorithm to
group putative orthologs and paralogs. In OrthoDB, Smith-Waterman BBH are
triangulated.

Most of the recent graph-based implementations have been used to establish
public databases of orthologs. These can be browsed online and the material can
be downloaded for scientific use. Of particular interest is OrthoDB, which
provides information on ortholog relationships at any level of the available
taxonomic tree. 

These methods arm scientists with powerful tools for comparing genomes, and with
next-generation, high-throughput sequencing technologies providing vast amounts
of data as well as the ever-decreasing cost of sequencing a complete
genome\footnote{National Human Genome Research Institute
(\url{http://www.genome.gov/sequencingcosts/}, last updated November 26, 2012)},
new discoveries in the fields of comparative genomics and phylogenetics are made
every day. 

