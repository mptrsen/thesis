\begin{figure}[t]
	\centering
	\def\svgwidth{0.8\textwidth}
	\input{img/graph-based-strat.pdf_tex}
	\caption[Graph-based strategy]{Graph-based approach to asses gene orthology. 
		\textbf{a)} In the evolutionary scenario of a gene family with two speciation
			events S\textsubscript{1} and S\textsubscript{2} and one intraspecific
			gene duplication event D\textsubscript{1}, the gene in the frog is
			orthologous to all other genes. The red and blue genes are orthologs among
			themselves, but paralogs to each other.
		\textbf{b)} The corresponding graph. Genes are represented as circles and
			evolutionary relationships by edges. While the red and blue genes form
			one-to-one relationships among themselves, the gene in the frog has a
			one-to-many relationship to the red and blue genes. A bidirectional best
			hit search strategy would detect only the highest scoring one of these
			pairs. 
		Graphic from \citet{altenhoff2012-1}.
	}
	\label{fig:graph-based-strat}
\end{figure}
