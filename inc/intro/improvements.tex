The \hamstr approach, applying a triangulation strategy with a joint application
of profile HMMs and BLAST, exhibits a very low false positive rate and good
sensitivity \citep{ebersberger2009}. Despite these demonstrated advantages of
the \hamstr algorithm, there are opportunities for improvement on this
implementation.

The most prominent issue, and a very practical one, is that of usability:
\hamstr in its original version is excessively hard to install and run
successfully for a user who is not proficient with a UNIX operating system and
Perl (Meusemann 2011, pers. comm.). The reason for this difficulty is twofold:
firstly, the Perl program code has to be edited prior to running the pipeline
and adjusted for the given environment. This is an error-prone practice and for
a non-technical user a major obstacle. The second reason for \hamstr's
inconvenience is the generation of custom ortholog sets. The files containing
the amino acid sequence data must be provided to the pipeline in a special
structure that is not trivial to set up manually, and frequently requires
scripting to automate repetitive tasks. An unexperienced user with no knowledge
of programming is not able to perform this, and as a consequence, is restrained
to ortholog sets that are provided by the
developers\footnote{\url{http://www.deep-phylogeny.org/hamstr/download/datasets/hmmer3/}}.

\hamstr makes use of many external tools such as \tool{grep} or \tool{sed} to
perform tasks such as identifying a specific sequence in a Fasta file or to
substitute patterns in a string. These tasks can (and should) be done using
built-in Perl functions. Starting a subprocess to call an external program adds
overhead and makes the pipeline slower as well as increase portability issues.

Genewise, which is used in \hamstr to predict open reading frames, cannot output
frameshift-corrected, corresponding nucleotide sequences for a given amino acid
sequence. Using the advanced Exonerate \citep{slater2005}, the successor to
Genewise, which is around 1,000 fold
faster\footnote{\url{http://archive.sciencewatch.com/dr/fmf/2008/08julfmf/08julfmfBirney/}},
is therefore advisable. Implementing Exonerate in a pipeline for orthology
prediction pipeline would, besides accelerating the process, enable output of
frameshift-corrected nucleotide sequence data.

Due to the algorithm that \hamstr uses, it is ``memoryless'', meaning it does
not cache whether a specific contig has already been mapped to an ortholog
cluster. As a result, it is possible that the same contig is mapped to two (or
more) different ortholog groups. This has happened in practice (Struck 2012,
pers. comm.), and can be considered a serious problem.

In addition to \hamstr being inconvenient for the average user, it can be argued
that using BLAST---which does not benefit from the accuracy of a HMM
profile---for the reciprocal search is a waste of specificity: HMM technology is
capable of detecting peptide sequences so remotely similar that BLAST would not
find them, therefore they would be discarded during the step in which the
ortholog status of these sequences is verified. If \hamstr were written in a
modular fashion, it would be easy to switch from BLAST to HMMER or other search
algorithms for the reciprocal search in order to test this hypothesis. 
