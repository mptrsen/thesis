Since the advent of DNA sequencing technology and the reconstruction of
genealogical relationships based on nucleic or amino acid sequences, the
challenge has arisen to select appropriate, comparable molecular characters for
phylogenetic analyses. So-called orthologs are the only type of molecular characters
that can be used as evidence of a speciation event. 

A number of techniques have been developed to find orthologs in genomes.
However, in a transcriptome, which is only the subset of a genome that is expressed at the
time of RNA preservation, these methods cannot be applied because of the inherent
incompleteness of transcriptomic data. In the present thesis, I outline the
concept  of gene orthology and infer a software pipeline that allows orthology
prediction in transcriptome data.

