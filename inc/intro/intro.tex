Since the advent of DNA sequencing technology and the reconstruction of
genealogical relationships based on nucleic or amino acid sequences, the
challenge has arisen to select appropriate, comparable molecular characters for
phylogenetic analyses. So-called orthologs are the only type of molecular
characters that can be used as evidence of a speciation event. 

A number of techniques has been developed to assess orthology in genomes. In
recent research, transcriptomes---which are only the subset of a genome that is
expressed at the time of RNA preservation---are frequently used because of lower
sequencing cost. However, methods that work in whole genomes cannot be applied
to transcriptomes because of the inherent incompleteness of transcriptomic data.
In the present thesis, I outline the concept of gene orthology and infer a
software pipeline that allows orthology prediction in transcriptome data.

