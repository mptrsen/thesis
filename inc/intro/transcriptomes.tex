Taking a step back and asking why genomes are studied, many times the answer is
that we want to discover the function of genes and proteins. The genome is made
up of long strands of deoxyribonucleic acid (DNA). They are paired in
chromosomes and contain the necessary instructions to create and maintain cells.
Proteins are generated according to information on the DNA in a two-step
process: during \emph{transcription}, messenger ribonucleic acid (mRNA) is
synthetisized, an intermediate product called \emph{transcript} that leaves the
nucleus. In the second step, \emph{translation}, the mRNA is used by specialized
organelles, namely ribosomes, to synthetisize proteins from amino acids. During
transcription, in a process called \emph{splicing}, non-coding regions,
so-called \emph{introns} of the mRNA are removed. The resulting molecule
contains only \emph{exons}, coding regions of the genome that are relevant for
translation. The ratio of exons to introns varies across organisms; e.g., the
human genome consists of only 1.1\% to 1.4\% exons and 24.4\% to 37.8\% introns,
the rest is intergenic DNA \citep{venter2001}. That means the majority of the
genome does not code for proteins, and the analysis of genomes for gene
functions requires screening the data for those regions, which adds further
challenges and complications.

The study of \emph{transcriptomes} circumvents these problems. Like a genome is
a collection of all the genes in a cell, a transcriptome is the collection of
all transcripts at the time of RNA preservation. Transcriptomes are also used in
phylogenetics and phylogenomics to reconstruct species trees based not only on
one or a few genes, but supermatrices of more than 1,000 genes, or more than
10,000 species \citep{beiko2011}. The combination of low-cost next-generation
sequencing and the intron-free nature of transcriptomes allow access to the same
gene concentration at a fraction of the cost for a complete genome. However,
next-generation sequencing assembly products are not only fragmented and possibly
redundant (overlapping), but most importantly they are \emph{incomplete} because
not all genes may have been expressed at the time of preservation. Due to this
inherent incompleteness of transcriptomes, orthology prediction approaches that
are used on complete genomes cannot be used.

Performing a bidirectional search on a complete genome as described above, it
can happen that two genes that are identified as mutually closest to each other
are in fact not orthologs, but paralogs where the corresponding orthologs have
been lost in both species. This situation is called \emph{differential gene
loss} and best solved using a tree-based approach. In the face of the
aforementioned difficulties of a tree-based strategy, some algorithms implement
\emph{triangulation}: To identify a gene as orthologous, it must be BBH not only
in two, but in three species. The third gene functions as ``witness of
non-orthology'' \citep{dessimoz2006}. A triangulation strategy can also be
applied to transcriptomes, the incompleteness of which may be seen as a similar
situation as potential gene loss in complete genomes. One implementation example
is \hamstr \citep{ebersberger2009}, on which I will elaborate in the following.

