Hidden Markov models (HMMs) are probabilistic models that are generally
applicable to ``linear'' problems and have been widely used in speech
recognition algorithms for thirty years. 

HMMs were introduced into computational biology by \cite{churchill1989} and
used as profile models since the 1990s (\cite{krogh1994}).

to time series or linear sequences.  This property makes them very useful for
application on biological sequences.

Proteins, RNAs and other biological molecular sequences can usually be
classified into families of related sequences and structures
(\cite{henikoff1997}).

In contrast to a sequence similarity-based search tool like BLAST,
Needleman-Wunsch (\cite{needleman1970}) or Smith-Waterman (\cite{smith1981}),
HMM-based alignment algorithms search for homology. 

\begin{figure}[h]
	\begin{center}
		\def\svgwidth{0.8\textwidth}
		\input{img/hmm-eddy.pdf_tex}
	\end{center}
	\caption[A simple hidden Markov model]{A simple hidden Markov model}
	\label{fig:hmm}
\end{figure}
