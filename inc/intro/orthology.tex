When reconstructing species lineages, so called homologous characters are used
to reconstruct ancestral trees. The term homolog was introduced in 1848
(\cite{owen1848}) and was used to describe ``the same organ in different animals
under different every variety of form and function''. Similarly, analogs were
defined as ``part or organ in one animal which has the same function as another
part or organ in a different animal''. At that time, Owen had no notion of the
concept of evolution, and in the famous \emph{Origin of Species}
(\cite{darwin1859}), the term homology is never used, but in a review
(\cite{owen1860}), he refers to homology as evidence of evolution.

A morphological character is the phenotypic reflection of a molecular character.
Since molecular genetics entered the field of phylogenetics during the 1960s,
these are used in numerous studies on species lineages.  Molecular characters,
such as genes, are homologous if they share a common origin. However, this is
not sufficient to infer reliable phylogenies based on molecular data. Genes can
be subject to a number of events in the course of their evolutionary history,
such as speciation, gene duplication, gene loss, horizontal gene transfer and
fusion, fission and other rearrangements of genes (\cite{koonin2005}). These
different types of relatedness between molecular sequences have made new
definitions necessary (see figure \ref{fig:orthology}).

\begin{figure}[h]
	\centering
	\def\svgwidth{0.8\textwidth}
	\input{img/orthology-paralogy.pdf_tex}
	\caption[Orthology, paralogy, and xenology]{Subtypes of homology. The red
		arrow denotes horizontal gene transfer; AB1 is \emph{xenologous} to all
		other genes. B1 and C1 are \emph{orthologs}. Both C2 and C3 are
		\emph{inparalogs} to each other but \emph{co-orthologs} to B2, as are B1 and
		C1 compared to A1. B1 and B2 are outparalogs. Graphic adapted from
		\citet{fitch2000}.
	}
	\label{fig:orthology}
\end{figure}


Homologous genes that are related by a speciation event are called
\emph{orthologs} (\cite{fitch1970}). They reflect species phylogeny directly and
are most commonly used to infer species lineages. \emph{Paralogous} genes are
also homologous, but they are related by a gene duplication event and are not
involved in horizontal radiation. Further distinction must be made among
paralogous genes (\cite{sonnhammer2002}): paralogous genes that are related by a
lineage-specific duplication are called \emph{outparalogs} if the duplication
occurred prior to a given speciation event, while genes that result from a
lineage-specific duplication subsequent to a given speciation event are called
\emph{inparalogs}.  These distinctions are important when looking at internal
branches of a phylogenetic tree.

The fourth subtype of homology that is called \emph{xenology}. It is defined as
the condition where the history of the gene involves horizontal, or
interspecies, gene transfer (\cite{gray1983}). This is the only form of homology
in which the gene lineage cannot be traced back to a parent, but instead from
one organism to another.

In molecular phylogenetics, ortholog characters are used to reconstruct species
lineages, because these sequences are the only homologous sequences whose
phylogeny reflects exactly the true phylogeny of the species from which the
sequences were obtained. Orthologs are also used to study mechanisms of gene and
genome evolution (\cite{dessimoz2012}). Orthologs tend to be more functionally
similar than paralogs (\cite{altenhoff2012}); this emphasizes the so-called
\emph{ortholog conjecture} (\cite{tatusov1997}) and is the reason why the
analysis of protein families often requires orthology among the investigated
genes.

