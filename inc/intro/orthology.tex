When reconstructing the evolution of species lineages, so called homologous
characters are used to reconstruct phylogenetic trees. The term homolog was
introduced by \citet{owen1848} and was used to describe ``the same organ in
different animals under different every variety of form and function''.
Similarly, analogs were defined as ``part or organ in one animal which has the
same function as another part or organ in a different animal''. At that time,
Owen had no notion of the concept of evolution, and in the famous \emph{Origin
of Species} \citep{darwin1859}, the term homology is never used. However, in a
review, \citet{owen1860} refers to homology as evidence of evolution.

A morphological character is the phenotypic reflection of genetic information.
Since the analysis of molecular data entered the field of phylogenetics during
the 1960s, these are used in numerous studies on species relationships.
Molecular characters, such as the DNA sequences of genes, are homologous if they
share a common origin. However, this is not sufficient to infer reliable
phylogenies based on molecular sequence data. Genes do not only dplicate during
a speciation event, but can be subject to a number of events in the course of
their evolutionary history, such as speciation, gene duplication, gene loss,
horizontal gene transfer as well as fusion, fission and other rearrangements of
genes \citep{koonin2005}.  These different types of relatedness between
sequences of molecular characters have made new definitions necessary (see
\autoref{fig:orthology}).

\begin{figure}[h]
	\centering
	\def\svgwidth{0.8\textwidth}
	\input{img/orthology-paralogy.pdf_tex}
	\caption[Orthology, paralogy, and xenology]{Subtypes of homology. The red
		arrow denotes horizontal gene transfer; AB1 is \emph{xenologous} to all
		other genes. B1 and C1 are \emph{orthologs}. Both C2 and C3 are
		\emph{inparalogs} to each other but \emph{co-orthologs} to B2, as are B1 and
		C1 compared to A1. B1 and B2 are outparalogs. Graphic adapted from
		\citet{fitch2000}.
	}
	\label{fig:orthology}
\end{figure}


Homologous genes in two or more species that are related by a speciation event
are called \emph{orthologs}\footnote{\emph{ortholog} n., \emph{orthologous}
adj.; the other terms are flexed accordingly.} \citep{fitch1970}. They reflect
species phylogeny directly and are most commonly used to infer phylogenetic
relationships between species. \emph{Paralogs} are also homologous genes, but
they are related by a gene duplication event within a species and are not
involved in horizontal radiation \citep{ohno1970}. Further distinction must be
made among paralogous genes \citep{remm2001}: paralogous genes that are related
by a lineage-specific duplication are called \emph{outparalogs} if the
duplication occurred prior to a given speciation event. On the contrary, genes
that result from a lineage-specific duplication subsequent to a given speciation
event are called \emph{inparalogs}. Additionally, two genes in a single species
that are paralogous to each other can be \emph{co-orthologs} to a gene in
another species. These distinctions are important when looking at internal
branches of a phylogenetic tree. 

A fourth subtype of homology is called \emph{xenology}. It is defined as the
condition in which the history of the genes involves horizontal, or
interspecies, gene transfer \citep{gray1983}. This is the only form of homology
in which the gene lineage cannot be traced back to a parent, but instead from
one organism to another.

In molecular phylogenetics, orthologous molecular sequences are used to
reconstruct genealogical relationships, because these sequences are the only
homologous sequences whose phylogeny reflects the genealogical relationships of
the species from which the sequences were obtained. Orthologs are also used to
study mechanisms of gene and genome evolution \citep{dessimoz2012}. Orthologs
tend to be more functionally similar than paralogs \citep{altenhoff2012}. This
is the so-called \emph{ortholog conjecture} \citep{tatusov1997}, and the reason
why the analysis of protein families often relies on orthology among the
investigated genes. In addition, housekeeping genes, i.e., genes that are
essential to keeping an organism alive, underlie stronger sequence conservation
due to selection pressure \citep{she2009} and are more likely to be orthologous
across species \citep{waterhouse2011}. 
