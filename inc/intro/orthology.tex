In phylogenies, so called homologous characters are used to reconstruct
ancestral trees. The term homolog was introduced in 1848 (\cite{owen1848}) and
was used to describe ``the same organ in different animals under different every
variety of form and function''. Similarly, analogs were defined as ``part or
organ in one animal which has the same function as another part or organ in a
different animal''. In the famous \emph{Origin of Species} (\cite{darwin1859}),
the term homology is never used, but in a review (\cite{owen1860}), homology is
referred to as evidence of evolution.

A morphological character is the phenotypic reflection of a molecular character.
Since the advent of molecular sequencing techniques and the reconstruction of
phylogenies based on nucleic or amino acid sequences, homology is still a required,
but no longer a sufficient criterion for comparing characters.

Molecular characters, such as genes, are homologous if they share a common
origin (\cite{koonin2005}). However, this is not sufficient to infer reliable
phylogenies based on molecular data. Genes can be subject to a number of events
in the course of their evolutionary history, such as speciation, gene
duplication, gene loss, horizontal gene transfer and fusion, fission and other
rearrangements of genes. These different types of relatedness between molecular
sequences have made new definitions necessary.

Homologous genes that are related by a speciation event are called
\emph{orthologs} (\cite{fitch1970}). They reflect species phylogeny directly and
are most commonly used to infer species lineages. \emph{Paralogous} genes are
also homologous, but they are related by a gene duplication event and are not
involved in horizontal radiation. Further distinction must be made among
paralogous genes (\cite{sonnhammer2002}): paralogous genes that are related by a
lineage-specific duplication are called \emph{outparalogs} if the duplication
occurred prior to a given speciation event, while genes that result from a
lineage-specific duplication subsequent to a given speciation event are called
\emph{inparalogs}.  These distinctions are important when looking at internal
branches of a phylogenetic tree.


\subsection{Definitions}

Definitions of:
\begin{itemize}
	\item orthologs
	\item paralogs
	\item inparalogs
	\item outparalogs
	\item xenologs?
\end{itemize}

Definitions done.

Strategies for orthology prediction:

Tree-based and graph-based approaches: advantages and pitfalls

\begin{description}
	\item[\cite{mirkin1995}] Tree-based approach to orthology prediction
	\item[\cite{yuan1998}] Tree-based approach to orthology prediction
	\item[\cite{kuzniar2008}] Review of approaches
\end{description}

Graph-based approaches:

Triangulation

\begin{itemize}
	\item OrthoDB 
	\item COG (KOG, EGO, etc)
\end{itemize}

Reciprocal best hit (RBH) 

\begin{itemize}
	\item InParanoid (BLAST based)
\end{itemize}

Markov clustering:

\begin{itemize}
	\item OrthoMCL
\end{itemize}

