HaMStR \citep{ebersberger2009} implements a graph-based approach using hidden
Markov models (HMMs, see section \ref{sec:hmms}). It is aimed specifically at
searching for orthologs in expressed sequence tag (EST) data, which is sequenced
using complementary DNA (cDNA) libraries. This cDNA is generated from mRNA and
therefore contains no introns. EST data can be redundant and fragmented, which
is why methods for orthology prediction in genomic data cannot be applied.

The HaMStR algorithm goes as follows:

\begin{enumerate}
	\item For each HMM, do the following:
		\begin{enumerate}
			\item Search the EST library. If matches were found, do the following for
				each:
			\begin{enumerate}
				\item Search the hit sequence against a BLAST database of all reference
					proteomes (the ``reciprocal BLAST''). 
				\item If matches were found: 

			\end{enumerate}
		\end{enumerate}
\end{enumerate}

\begin{figure}[h]
	\centering
	\def\svgwidth{0.8\textwidth}
	\input{img/triangulation.pdf_tex}
	\caption[Triangulation]{Triangulation:
		\begin{inparaenum}
			\item The transcript sequence space is searched using a hidden Markov
				model (HMM), which is a statistical representation of the reference
				sequences that were used to build it.
			\item A reference proteome is searched using the match sequence from the
				HMM search (reciprocal search).
			\item If the reciprocal search match sequence is one that was used to
				build the HMM, then orthology is assumed.
		\end{inparaenum}
	}
	\label{fig:hamstr}
\end{figure}

% enable this if you need it, it makes compilation a bit slower
%\usetikzlibrary{shapes,arrows}
\tikzstyle{rect} = [
	rectangle,
	draw,
	fill = green!15,
	text width = 9 em,
	text centered,
	minimum height = 3 em,
]
\tikzstyle{block} = [
	rectangle,
	draw,
	fill = blue!20,
	text width = 13em,
	text centered,
	rounded corners,
	minimum height = 3em,
]
\tikzstyle{decision} = [
	diamond,
	draw, 
	fill = blue!20, 
	text width = 4.5em,
	text centered,
	node distance = 14em,
	inner sep = 0pt
]
\tikzstyle{line} = [draw, -latex]
\tikzstyle{cloud} = [
	draw,
	ellipse,
	fill = red!20,
	text width = 7.5em,
	text badly centered,
	node distance = 3cm,
	minimum height = 2em
]

\hyphenation{ref-e-ren-ce tran-scrip-to-me or-tho-lo-gous}
\begin{tikzpicture}[node distance = 2cm, auto]
	% nodes
	\node[rect] (transcriptome) {Transcriptome library};
	\node[block, below of = transcriptome] (hmmsearch) {Search transcriptome library using next HMM};
	\node[rect, left of = hmmsearch, node distance = 14em] (orthologs) {pHMMs of orthologous sequences};
	\node[block, above of = orthologs] (orthodb) {Orthologous sequences of reference taxa from OrthoDB};
	\node[block, below of = hmmsearch, node distance = 4em] (hmmhits) {BLAST result(s) against next reference taxon};
	\node[decision, below of = hmmhits, node distance = 7em] (blast) {BLAST hit in HMM?};
	\node[rect, right of = hmmhits, node distance = 14em] (proteomes) {Proteomes of the reference taxa};
	\node[cloud, below of = blast, node distance=7em] (orthologous) {Orthologous, save \& process};
	\node[decision, below of = orthologous, node distance = 6em] (hmmsleft) {HMMs left?};
	\node[decision, left of = blast, node distance = 16em] (reftaxaleft) {Reference taxa left?};
	\node[cloud, below of = reftaxaleft, node distance = 7em] (notorthologous) {Not orthologous, discard};
	\node[draw, below of = hmmsleft, node distance = 6em] (end) {End};

	% lines
	\path[line](transcriptome) -- (hmmsearch);
	\path[line](orthodb) -- (orthologs);
	\path[line](orthologs) -- (hmmsearch);
	\path[line](hmmsearch) -- (hmmhits);
	\path[line](proteomes) -- (hmmhits);
	\path[line](hmmhits) -- (blast);
	\path[line](blast) -- node[near start]{no} (reftaxaleft);
	\path[line](blast) -- node[near start]{yes} (orthologous);
	\path[line](orthologous) -- (hmmsleft);
	\path[line](hmmsleft) -| node[near start]{yes} ([xshift = 18.0em]hmmsleft.east) |- (hmmsearch);
	\path[line](reftaxaleft) |- node[near start]{yes} (hmmhits);
	\path[line](reftaxaleft) -- node[near start]{no} (notorthologous);
	\path[line](notorthologous) |- (hmmsleft);
	\path[line](hmmsleft) -- node[near start] {no} (end);
\end{tikzpicture}

