\hamstr implements a graph-based approach using profile hidden Markov models
(pHMMs, explained in section \ref{sec:hmms}). It is aimed specifically at
searching for orthologs in \emph{expressed sequence tag} (EST)
data,\footnote{National Center for Biotechnology Information: EST fact sheet
(\url{http://www.ncbi.nlm.nih.gov/About/primer/est.html}, revised: March 29,
2004)} which are sequenced using complementary DNA (cDNA) libraries in a process
called \emph{shotgun sequencing}. The cDNA is cloned from mRNA and therefore
contains no introns, and, like a transcriptome, EST data is incomplete, i.e., it
contains only nucleotide sequences of the genes that were expressed at the time
of RNA preservation. Due to the way it is sequenced and assembled, EST data can
be redundant and fragmented as well, which is why methods for orthology prediction in
genomic data cannot be applied, and approaches like described above have to be
used.

\hamstr uses a three-step triangulation strategy as depicted in
\autoref{fig:hamstr-triangulation}: Using a HMM that represents amino acid
sequences that are known to be orthologous, the EST sequence space is
searched for matches (\autoref{fig:hamstr-triangulation}a). To verify orthology
in the HMM hits, each hit sequence is compared via BLASTP \citep{altschul1997}
to a so-called reference taxon (\autoref{fig:hamstr-triangulation}b). Note that
the proteome, itself being complete, contains the amino acid sequences that are
known to be orthologous. If the BLAST search hits a sequence that was used to
generate the HMM (\autoref{fig:hamstr-triangulation}c), then orthology is assumed
and the EST in question is added to the ortholog cluster. Otherwise, it is
discarded.

\begin{figure}[t]
	\centering
	\def\svgwidth{0.8\textwidth}
	\input{img/triangulation.pdf_tex}
	\caption[Triangulation in \hamstr]{Triangulation in \hamstr:
		\textbf{a)} The transcript sequence space is searched using a hidden Markov
			model (HMM), which is a statistical representation of the orthologous
			reference sequences that were used to build it.
		\textbf{b)} A reference proteome is searched using the match sequence from
			the HMM search (reciprocal search).
		\textbf{c)} If the reciprocal search match sequence is one that was used to
			build the HMM, then orthology is assumed.
	}
	\label{fig:hamstr-triangulation}
\end{figure}




% enable this if you need it, it makes compilation a bit slower
%\usetikzlibrary{shapes,arrows}
\tikzstyle{rect} = [
	rectangle,
	draw,
	fill = green!15,
	text width = 9 em,
	text centered,
	minimum height = 3 em,
]
\tikzstyle{block} = [
	rectangle,
	draw,
	fill = blue!20,
	text width = 13em,
	text centered,
	rounded corners,
	minimum height = 3em,
]
\tikzstyle{decision} = [
	diamond,
	draw, 
	fill = blue!20, 
	text width = 4.5em,
	text centered,
	node distance = 14em,
	inner sep = 0pt
]
\tikzstyle{line} = [draw, -latex]
\tikzstyle{cloud} = [
	draw,
	ellipse,
	fill = red!20,
	text width = 7.5em,
	text badly centered,
	node distance = 3cm,
	minimum height = 2em
]

\hyphenation{ref-e-ren-ce tran-scrip-to-me or-tho-lo-gous}
\begin{tikzpicture}[node distance = 2cm, auto]
	% nodes
	\node[rect] (transcriptome) {Transcriptome library};
	\node[block, below of = transcriptome] (hmmsearch) {Search transcriptome library using next HMM};
	\node[rect, left of = hmmsearch, node distance = 14em] (orthologs) {pHMMs of orthologous sequences};
	\node[block, above of = orthologs] (orthodb) {Orthologous sequences of reference taxa from OrthoDB};
	\node[block, below of = hmmsearch, node distance = 4em] (hmmhits) {BLAST result(s) against next reference taxon};
	\node[decision, below of = hmmhits, node distance = 7em] (blast) {BLAST hit in HMM?};
	\node[rect, right of = hmmhits, node distance = 14em] (proteomes) {Proteomes of the reference taxa};
	\node[cloud, below of = blast, node distance=7em] (orthologous) {Orthologous, save \& process};
	\node[decision, below of = orthologous, node distance = 6em] (hmmsleft) {HMMs left?};
	\node[decision, left of = blast, node distance = 16em] (reftaxaleft) {Reference taxa left?};
	\node[cloud, below of = reftaxaleft, node distance = 7em] (notorthologous) {Not orthologous, discard};
	\node[draw, below of = hmmsleft, node distance = 6em] (end) {End};

	% lines
	\path[line](transcriptome) -- (hmmsearch);
	\path[line](orthodb) -- (orthologs);
	\path[line](orthologs) -- (hmmsearch);
	\path[line](hmmsearch) -- (hmmhits);
	\path[line](proteomes) -- (hmmhits);
	\path[line](hmmhits) -- (blast);
	\path[line](blast) -- node[near start]{no} (reftaxaleft);
	\path[line](blast) -- node[near start]{yes} (orthologous);
	\path[line](orthologous) -- (hmmsleft);
	\path[line](hmmsleft) -| node[near start]{yes} ([xshift = 18.0em]hmmsleft.east) |- (hmmsearch);
	\path[line](reftaxaleft) |- node[near start]{yes} (hmmhits);
	\path[line](reftaxaleft) -- node[near start]{no} (notorthologous);
	\path[line](notorthologous) |- (hmmsleft);
	\path[line](hmmsleft) -- node[near start] {no} (end);
\end{tikzpicture}


