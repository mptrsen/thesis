\label{sec:mysql-performance}
As outlined in \autoref{sec:mysql}, MySQL was used for various reasons, one of
which is that a RDBMS is optimized for speed and efficiency. This is true
especially for small databases with tables that contain less than five million
records in this schema (see \autoref{sec:database-structure}). However, in its
current structure, without administrative access to MySQL server
variables---which allow more fine-tuning for performance
\citep{schwartz2012}---it does not scale well: above of 5 million records,
MyISAM performance for re-indexing the table after uploading new transcriptome
sequence data starts to drop noticeably (see \autoref{fig:transaction-time}). 
This problem can be circumvented by redesigning the database structure. A
possible solution is discussed in \autoref{sec:table-per-species}.

\begin{figure}[h]
\centering
\def\svgwidth{\textwidth}
\input{img/transaction-time.pdf_tex}
\caption{Transaction time}
\label{fig:transaction-time}
\end{figure}


