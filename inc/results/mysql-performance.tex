As outlined in \autoref{sec:mysql}, MySQL was used for various reasons, one of
which is that a DBMS is optimized for speed and efficiency. This is true
especially for small databases with tables that contain less than five million
records in this schema (see \autoref{sec:database-structure}). Above of 5
million records, MyISAM performance for re-indexing after uploading new
transcriptome sequence data starts to drop noticeably (see
\autoref{fig:transaction-time}). A solution to this issue is to partition the
tables ``ests'', ``hmmsearch'', and ``blast''. 

\begin{figure}[h]
\centering
\def\svgwidth{0.8\textwidth}
\input{img/transaction-time.pdf_tex}
\caption[Re-indexing performance on MyISAM tables]{
	Re-indexing performance on MyISAM tables. Starting with about $5.5\times10^6$
	rows in the table ``ests'', MyISAM re-indexing performance starts to drop.
	The indices had been disabled prior to data upload in order to improve upload
	performance. A solution to this problem is to split up the table ``ests'' into
	separate tables for each query taxon.
}
\label{fig:transaction-time}
\end{figure}


During testing, it became obvious that not only the indexing strategy and the
query design, but also the database structure plays a major part in terms of
performance. Especially the tables ``ests'', ``hmmsearch'' and ``blast'' become
excessively large after a few analyses: assume a number of 500,000 transcript
sequences that are uploaded and an ortholog set of 4,000 OGs. If each of the
4,000 HMM searches obtains an average of 20 hits during the HMM search, and each
of the 20 BLAST searches obtain an average of 50 hits, this amounts to $4,000
\cdot 20 = 8 \cdot 10^4$ rows in the table ``hmmsearch'' and $4 \cdot 10^6$ rows in the
table ``blast''. The rows themselves do not contain much data (131 B
on average), but their sheer number brings InnoDB performance to its knees.
Insertion of new data does not become noticably slow, but the deletion of
unneeded records in the event of a repeated analysis that clears previous
results from the same query species from the database prior to the actual
analysis. The InnoDB cluster index physically orders the table based on the
primary key or the first unique key it can utilize. When one row is removed, the
entire table is reordered on disk for speed and defragmentation. With increasing
table size, this operation takes exponentially long (see
\autoref{fig:transaction-time}).

From what I have learned, the simplest and most performant solution to this
problem is to create individual tables for each query taxon. Dropping or
truncating (emptying) a table and recreating it is much faster than deleting a
large portion out of a huge table. 
