As outlined in \autoref{sec:mysql}, MySQL was used for various reasons, one of
which is that a DBMS is optimized for speed and efficiency. This is true
especially for small databases with tables that contain less than five million
records in this schema (see \autoref{sec:database-structure}). However, in its
current structure, without administrative access to MySQL server
variables---which allow more fine-tuning for performance
\citep{schwartz2012}---it does not scale well: above of 5 million records,
MyISAM performance for re-indexing after uploading new transcriptome sequence
data starts to drop noticeably (see \autoref{fig:transaction-time}). A solution
to this issue is to partition the tables ``ests'', ``hmmsearch'', and ``blast''
into individual tables for each query dataset. Transcriptome sequence data as
generated within the 1KITE project has a number of sequences in the order of
$5\cdot10^5$, which amounts to the same number of records in the database. That
means the database in its current form and without administrative access to
server variables can only store up to about ten query datasets before it becomes
slow.

\begin{figure}[h]
\centering
\def\svgwidth{0.8\textwidth}
\input{img/transaction-time.pdf_tex}
\caption[Re-indexing performance on MyISAM tables]{
	Re-indexing performance on MyISAM tables. Starting with about $5.5\times10^6$
	rows in the table ``ests'', MyISAM re-indexing performance starts to drop.
	The indices had been disabled prior to data upload in order to improve upload
	performance. A solution to this problem is to split up the table ``ests'' into
	separate tables for each query taxon.
}
\label{fig:transaction-time}
\end{figure}


During testing, it became obvious that not only the indexing strategy and the
query design, but also the database structure plays a major part in terms of
performance. Especially the tables ``hmmsearch'' and ``blast'' become
excessively large after a few analyses: with an ortholog set of 4,000 OGs, if
each of the 4,000 HMM searches obtains an average of 20 hits during the HMM
search, and each of the 20 BLAST searches obtain an average of 50 hits, this
amounts to $4,000 \cdot 20 = 8 \cdot 10^4$ rows in the table ``hmmsearch'' and
$4 \cdot 10^6$ rows in the table ``blast''. The rows themselves do not contain
much data (only 131 \nomenclature{B}{Byte} on average), but due to the large
number of records, InnoDB performance with the current schema does not scale
well. Insertion of new data does not become noticably slow; however, deleting
old analysis results takes very long. The InnoDB cluster index physically orders
the table based on the primary key or the first unique key it can utilize. When
one row is removed, the entire table is reordered on disk for speed and
defragmentation.  With increasing table size, this operation takes exponentially
long (see \todo{data!}).

The simplest and most performant solution to this problem is to create
individual tables for each query taxon. Dropping or truncating (emptying) a
table and recreating it is much faster than deleting a large portion out of a
huge table, especially when there are indices involved.

Especially the InnoDB engine
benefits significantly from large amounts of RAM \citep{schwartz2012} because
it can cache large portions of a table in RAM and thus does not need to access
the hard drive frequently.
