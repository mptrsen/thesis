\label{sec:algorithm-reporting}
The evaluation program \tool{orthograph-reporter} assesses gene orthology among
the transcript sequence data in the database on the basis of the bidirectional
search results that were obtained using \tool{orthograph-analyzer}. Note that
this step is intended to be independent of the search algorithm and does not
modify the results that are already present in the database, leaving them in a
consistent state for evaluation under different criteria.

For generating the graph, \tool{orthograph-reporter} organizes the data from the
database in a special, multidimensional structure that is easy to parse:

\begin{lstlisting}[label={lst:structure},caption={Data structure for parsing the triangular
relationships between ortholog groups and transcripts}]
hmmsearch_evalue => {
	ortholog_group_id => [
		reciprocal_hit,
		reciprocal_hit,
	],
	ortholog_group_id => [
		reciprocal_hit,
		reciprocal_hit,
	],
	. . .
},
. . .
\end{lstlisting}

Each \emph{reciprocal\_hit} represents a BLAST search with the transcript
sequence as query in the proteome database for the ortholog set that was used.
The target can be compared to a list of amino acid sequences that were used in
constructing the HMM for the given \emph{ortholog\_group\_id}. If the reciprocal
search hit a matching amino acid sequence, this pair of [ortholog group +
transcript sequence] is considered orthologous.

The reporting algorithm is as follows (see also \autoref{fig:orthograph-graph}):

\begin{enumerate}
	\item Get configuration, initialize global variables.
	\item Get a list of ortholog group IDs and their associated ortholog sequences
		from the database.
	\item Get all search results from the database.
	\item Sort the \tool{hmmsearch} e-values in ascending order.
	\item Starting with the lowest \tool{hmmsearch} e-value, do the following:
	\begin{enumerate}
		\item For each ortholog group that has obtained a \tool{hmmsearch} hit with this
			e-value, do the following:
			\begin{enumerate}
				\item Sort the reciprocal hits by BLAST e-value in ascending order.
				\item For each reciprocal hit, do the following:
				\begin{enumerate}
					\item If the reciprocal search for this transcript hit a sequence that
						is in this ortholog group, and this transcript has not been assigned
						previously, then this is assumed to be a valid match. Continue.
						Otherwise, skip to the next reciprocal hit unless the ``soft
						threshold'' has been reached (in that case, skip to the next
						ortholog group).
					\item If the transcript does not overlap with an existing assignment,
						assign this transcript to this ortholog group. This transcript cannot
						be assigned again. 
				\end{enumerate}
			\item If there is a gap between the transcripts (the end of one transcript
				lies more than 1 bp before the start of the next), fill the gap with
				'\nomenclature{X}{Ambiguity character in amino acid sequences; may be
				any amino acid}' and concatenate the fragments. This ortholog group can also not be
				assigned again.
			\item Correct for frameshift errors and infer the coding sequence using
				Exonerate (not yet implemented at the time of this writing).
		\end{enumerate}
	\end{enumerate}
\end{enumerate}

The results are written to Fasta files---one for each ortholog group with the
transcript sequences that were mapped to it---and a summary table. The presence
of corresponding protein-coding nucleotide sequence data for the reference
proteomes in the database allows output of the corresponding nucleotide
sequences along with the amino acid sequences.

\begin{figure}[h]
	\centering
	\def\svgwidth{0.8\textwidth}
	\input{img/orthograph_graph.pdf_tex}
	\caption[Non-redundant assignment of ortholog relationships]{
		Non-redundant assignment of ortholog relationships. The ortholog groups (OG)
		are connected to the candidate hit transcripts by the e-value of the HMM
		search. Traversing the graph by ascending e-value and assigning the
		transcripts to the OG with the best hit results in transcript 3 being
		assigned to OG B, transcript 4 to OG C and transcript 2 to OG A. Transcript
		1 remains unassigned because the \tool{hmmsearch} e-value for OG A is higher than
		the e-value for transcript 2.
	}
	\label{fig:orthograph-graph}
\end{figure}



\begin{figure}[ht]
	\centering
	\def\svgwidth{0.8\textwidth}
	\pname is a graph-based approach to orthology prediction. The graph draws
ortholog relationships between transcripts and known orthologous groups (figure
\ref{fig:graph}).

\begin{figure}[ht]
	\label{fig:graph}
	\begin{center}
		\def\svgwidth{0.8\textwidth}
		\pname is a graph-based approach to orthology prediction. The graph draws
ortholog relationships between transcripts and known orthologous groups (figure
\ref{fig:graph}).

\begin{figure}[ht]
	\label{fig:graph}
	\begin{center}
		\def\svgwidth{0.8\textwidth}
		\input{img/graph.pdf_tex}
	\end{center}
	\caption[Graph-based approach to orthology prediction]{A two-dimensional
		graph. Transcript sequences from a large sequence space are assigned to
		ortholog groups (in circles). Since the transcript sequences are fragmented,
		multiple transcripts may be assigned to a single ortholog group.}
\end{figure}

	\end{center}
	\caption[Graph-based approach to orthology prediction]{A two-dimensional
		graph. Transcript sequences from a large sequence space are assigned to
		ortholog groups (in circles). Since the transcript sequences are fragmented,
		multiple transcripts may be assigned to a single ortholog group.}
\end{figure}

	\caption[Graph-based approach to orthology prediction]{
		Transcript sequences from a large sequence space are mapped to ortholog
		groups (OG, in circles). Since the transcript sequences may be fragmented,
		multiple transcripts may extend a single ortholog cluster.
	}
	\label{fig:graph}
\end{figure}




