The reporting algorithm is as follows:

\begin{enumerate}
	\item Get configuration, initialize global variables.
	\item Get a list of ortholog group IDs and their associated ortholog sequences
		from the database.
	\item Get all results in the form:
	\begin{lstlisting}
	hmmsearch_evalue => {
		ortholog_group_id => [
			reciprocal_hit,
			reciprocal_hit,
		],
		ortholog_gene_id => [
			reciprocal_hit,
			reciprocal_hit,
		],
	}
	\end{lstlisting}
	\item Sort the e-values in ascending order.
	\item Starting with the lowest hmmsearch e-value, do the following:
	\begin{enumerate}
		\item For each ortholog group that has a hit with this e-value, do the following:
		\begin{enumerate}
			\item Check whether this ortholog group is present in the list that was
				generated in step 2. If not, skip to the next group (this ortholog group
				has already been assigned a transcript).
			\item Sort the reciprocal hits by BLAST e-value in ascending order.
			\item For each reciprocal hit, do the following:
			\begin{enumerate}
				\item If the reciprocal search hit a sequence that is in this ortholog
					group, and this transcript has not been assigned previously, then this
					is a valid match. Otherwise, skip to the next reciprocal hit unless
					the ``soft threshold'' has been reached (in that case, skip to the
					next ortholog group).
				\item Assign this transcript to this ortholog group if it does not
					overlap with an existing assignment. This transcript cannot be
					assigned again.
			\end{enumerate}
			\item If there is a gap between the transcripts (the end of one transcript
				lies more than 1 bp before the start of the next), fill the gap with
				'X' and concatenate the fragments.
		\end{enumerate}
	\end{enumerate}
\end{enumerate}

