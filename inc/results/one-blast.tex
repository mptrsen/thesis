In \hamstr, one BLAST database had to be maintained for each reference taxon.
This led to the problem of BLAST scores and e-values not being normalized: since
the \hamstr BLAST databases are of differing sizes, the e-values, which depend
on the database size, cannot be compared across reference taxa. To generate a
``normalized'' e-value, \hamstr took an additional pairwise alignment step using
ClustalW.

\pname uses one BLAST database comprising all reference taxa proteomes. E-values
are comparable across taxa since the database is of constant size. Additionally,
only one BLAST search is required instead of $n$ searches, where $n$ is the
number of reference taxa, leading to a performance boost of up to $n-1$.
