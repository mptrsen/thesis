The graph-based approach to orthology prediction is in principle a simple
algorithm and easy to implement, but in practice offers not only technical
challenges when applied on a large scale such as on the 1,000 transcriptome
sequence data for the 1KITE\footnote{\url{http://1kite.org}} project, but also
numerous opportunities for extension and improvement. 

The results of this thesis are contained in the code I have written for this
project. For the present thesis, I have developed a pipeline called \pname
(\pfullname). It is a rewrite of the \hamstr approach using state-of-the-art
technology and programming techniques. Like in \hamstr, orthologs are clustered
on a two-dimensional graph based on triangular relationships (see
\autoref{sec:graph}), but the algorithm is different and, among other features,
disallows redundant assignment of transcript to different ortholog groups.

\pname is licensed under the GPL. The source code is hosted on GitHub at
\url{https://github.com/mptrsen/Orthograph}. This platform allows code version
synchronization using the revision control software Git. If Git is installed on
the computer, installation of \pname is simple: the user only needs to execute
\code{git fetch https://github.com/mptrsen/Orthograph.git} once. Subsequent
updates to the code repository can be synchronized by executing \code{git pull}
in the \pname directory. If Git is not available, the package can be downloaded
from GitHub as a zip archive and unpacked normally.

The most outstanding novelty in comparison to \hamstr is the use of the
relational database management system MySQL. It enables the program to
efficiently store and retrieve data as well as---with complex \code{JOIN}
queries---establish ortholog relationships. Additionally, the server-client
model of MySQL facilitates analyses on networked computer systems and, if
configured correctly, allows for high performance.

\pname is user-friendly in that it is mainly configuration file driven instead
of only accepting options on the command line. This allows for clear,
reproducible and easily automatable analyses. However, all options may also be
set on the command line, and will fall back to default values if left
unspecified.

The pipeline does not make use of external programs where avoidable, i.e., it
uses Perl built-in functionality for, e.g., pattern substitution. Subprocesses
are only started where the search programs are called, i.e., hmmsearch, BLASTP
and Exonerate. For MySQL, the internal driver and native Perl interface are
used.

In the following sections, I will explain the algorithm that \pname employs, the
database structure it uses, and how the graph of ortholog relationships is
constructed. During the development of \pname, I was faced with challenges of both
theoretical and practical nature, and I will also report on the techniques that
I implemented to overcome these obstacles.
