The graph-based approach to orthology prediction is in principle a simple
algorithm and easy to implement in a program. However, in practice, it offers
not only technical challenges when applied on a large scale, such as on the
transcriptome sequence data from 1,000 species that are analyzed within the
1KITE\footnote{\url{http://1kite.org}} project, but also numerous opportunities
for extension and improvement. 

For the present thesis, I developed a pipeline called \pname and used
state-of-the-art technology and programming techniques. Like in \hamstr,
orthologous amino acid or nucleotide sequences are clustered on a
two-dimensional graph (see \autoref{fig:graph}) based on triangular
relationships, but the algorithm in \pname is different and, among other
features, precludes redundant assignment of transcripts to different ortholog
groups.

\begin{figure}[ht]
	\centering
	\def\svgwidth{0.8\textwidth}
	\pname is a graph-based approach to orthology prediction. The graph draws
ortholog relationships between transcripts and known orthologous groups (figure
\ref{fig:graph}).

\begin{figure}[ht]
	\begin{center}
		\def\svgwidth{0.8\textwidth}
		\pname is a graph-based approach to orthology prediction. The graph draws
ortholog relationships between transcripts and known orthologous groups (figure
\ref{fig:graph}).

\begin{figure}[ht]
	\begin{center}
		\def\svgwidth{0.8\textwidth}
		\input{img/graph.pdf_tex}
	\end{center}
	\caption[Graph-based approach to orthology prediction]{A two-dimensional
		graph. Transcript sequences from a large sequence space are assigned to
		ortholog groups (in circles). Since the transcript sequences are fragmented,
		multiple transcripts may be assigned to a single ortholog group.}
	\label{fig:graph}
\end{figure}

	\end{center}
	\caption[Graph-based approach to orthology prediction]{A two-dimensional
		graph. Transcript sequences from a large sequence space are assigned to
		ortholog groups (in circles). Since the transcript sequences are fragmented,
		multiple transcripts may be assigned to a single ortholog group.}
	\label{fig:graph}
\end{figure}

	\caption[Graph-based approach to orthology prediction]{
		Transcript sequences from a large sequence space are mapped to ortholog
		groups (OG, in circles). Since the transcript sequences may be fragmented,
		multiple transcripts may extend a single ortholog cluster.
	}
	\label{fig:graph}
\end{figure}




The program has been designed to be platform-independent and has been developed
and tested on Linux systems. In theory, the pipeline runs in Windows
environments as well, however, the installation of the packages it depends on
(HMMER3, BLAST, Exonerate, and the MySQL driver for Perl) may be problematic on
Windows systems. This has not been tested yet.

\pname is licensed under the GPL. The source code is hosted on
GitHub\footnote{\url{https://github.com/mptrsen/Orthograph}}. This platform
allows code version synchronization using the revision control software Git. If
Git is present on a user's computer, the installation of \pname is simple: the
user only needs to execute \code{git clone
https://github.com/mptrsen/Orthograph.git} once. Git will create a directory and
download the source code to the local hard drive. Since the pipeline is written
in Perl, no compilation is necessary. Subsequent updates to the code repository
can be synchronized by executing \code{git pull} in the \pname directory. If Git
is not available, the package can be downloaded from GitHub as a zip archive and
unpacked normally.

The most outstanding novelty in comparison to \hamstr is the use of the
relational database management system MySQL. It enables the program to
efficiently store and retrieve data as well as---with complex SQL
queries---establish ortholog relationships. Additionally, the server-client
model of MySQL facilitates analyses on networked computer systems and allows for
high performance.

\pname is user-friendly in that it is mainly configuration file-driven instead
of only accepting options on the command line. This allows for clear,
reproducible, and easily automatable analyses. All options may also be set on
the command line and the program applies default values if none are explicitly
specified. This enables implementation of \pname in a larger pipeline.
Convenience of use is further enhanced by automatically generating an ortholog
set from provided information.

The pipeline makes use of as few external programs as possible. It
uses Perl built-in functionality for, \eg, pattern substitution. Subprocesses
are only started if the following search programs are required: \tool{hmmsearch},
BLAST and Exonerate. For interactions with the MySQL database, the MySQL driver
DBD for Perl and the native Perl database
interface DBI are used.

In the following sections, I explain the algorithm that \pname employs, the
database structure it relies on, and how the graph of ortholog relationships is
constructed. During the development of \pname, I was faced with challenges of
both theoretical and practical nature, and I also report on the techniques that
I implemented to overcome these obstacles.

