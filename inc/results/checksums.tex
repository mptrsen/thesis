Unique sequence IDs are necessary in order for \code{hmmsearch} not to create
confusion by treating whitespace in sequence headers as a description
separator. To avoid this, and to maintain a consistent naming scheme across
applications, \pname~uses a SHA-256 checksum to generate a unique ID for every
sequence. The checksum is generated using both the original header and the
sequence. Sequences are loaded into the database along with these checksums.
During the analysis, wherever a file is generated that includes sequence
identifiers, this checksum is used. This also eliminates the problem with
\code{fastatranslate} introducing whitespace that might confuse
\code{hmmsearch}.

It must be guaranteed that no two checksums, i.e., two sequence identifiers, are
ever the same. The SHA-256 hashing algorithm generates a checksum that is 160 bits,
or 40 hexadecimal characters in length. The probability $p$ of a hash collision
(i.e., two hashed elements returning the same checksum) in $n$ elements is

\begin{equation}
p \ge \frac{n (n-1)}{2} \times \frac{1}{2^b}
\label{eq:hashcollision}
\end{equation}

where $b$ is the number of bits generated by the hash function. There need to be
more than $1.7 \times 10^{15}$ objects for the SHA-256 hashes to exceed a collision
probability of $10^{-18}$. Since the hash space is expected to contain only a
number of objects in the range of $10^6$ to $10^{12}$, it is statistically safe
to assume that every checksum is unique. 


