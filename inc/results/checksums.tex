\label{sec:digests}
\tool{Hmmsearch} treats whitespace in a sequence identifier in a special way: it
splits the identifier at the first whitespace character and considers the part
of the string that comes after the whitespace as ``description of target''. As a
result of this practice, the actual target identifiers are no longer unique if
multiple sequence identifiers have identical strings before the first
whitespace.

Using the auto-incremented primary key integer from the database as a sequence
identifier is not feasible because it differs if an analysis is restarted and
the sequences are uploaded again: they are assigned different numbers. 

In order to uniquely identify sequences, and to maintain a consistent naming
scheme across applications, \pname uses a SHA-256 digest \citep{gallagher2008}
to generate a unique identifier for every sequence. The digest is generated
using both the original header and the sequence. Sequences are loaded into the
database along with these digests. During the analysis, wherever a file is
generated that requires sequence identifiers, this digest is used. 

It must be guaranteed that no two digests, \ie, two sequence identifiers, are
ever the same. The SHA-256 hashing algorithm generates a digest that is 160 bits,
or 40 hexadecimal characters in length. The probability $p$ of a hash collision
(\ie, two hashed elements returning the same digest) in $n$ elements is

\begin{equation}
p \ge \frac{n (n-1)}{2} \times \frac{1}{2^b}
\label{eq:hashcollision}
\end{equation}

where $b$ is the number of bits generated by the hash function. There need to be
more than $1.7 \times 10^{15}$ objects for the SHA-256 hashes to exceed a collision
probability of $10^{-18}$. Since the hash space is expected to contain only a
number of objects in the range of $10^6$ to $10^{12}$, it is statistically safe
to assume that every digest is unique. 


