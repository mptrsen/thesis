The analysis, \ie, the searching algorithm is as follows:

\begin{enumerate}
	\item Get configuration, initialize global variables.
	\item Check the following:
	\begin{enumerate}
		\item Do the user settings make sense? If not, exit.
		\item Are the paths to the input file and the programs correct? If not,
			exit.
		\item Do the output directories exist? If not, create them.
		\item Does the database structure exist? If not, exit.
	\end{enumerate}
	\item Backup old results, if requested.
	\item Clear old results from file system and database, if requested.
	\item Create the HMMs, if they do not exist.
	\item Create the BLAST database of all core taxa, if it does not exist.
	\item Translate the input file in all six reading frames, if it is nucleic
		acid data.
	\item Load the translated sequences into the database, generating unique
		SHA256 digests (see \autoref{sec:checksums}) on the way.
	\item Get all translated sequences of the target species from the database.
		Sequence identifiers are now SHA256 digests. Write the sequences into a
		temporary file.
	\item For each HMM, do the following:
	\begin{enumerate}
		\item Search the translated sequences in the temporary file with this HMM.
			Skip to the next HMM if no hits were obtained. Otherwise, read the tabular
			search report and store it in the database. 
		\item For each hit, do the following:
		\begin{enumerate}
			\item Get the hit sequence from the database and write the relevant
				(sub)sequence to a Fasta file. This information has been gained from the
				\tool{hmmsearch} report file.
			\item Search this sequence against the BLAST database. Skip to the next
				HMM if no hits were obtained. Otherwise, read the tabular search report
				and store it in the database. 
		\end{enumerate}
	\end{enumerate}
\end{enumerate}

Note that the analysis algorithm does not attempt to infer ortholog
relationships. Only the bidirectional searches are performed during this step. 
