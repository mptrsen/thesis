The server-client model of MySQL enables a distributed analysis on a computer
network. This is advantageous on either a HPC cluster designed for massively
parallelized computing, where many physical computers are connected in a
low-latency network and communicate with each other, or a network of standard
desktop computers, where the clients communicate with the server only. In either
environment, one computer acts (or multiple computers act: in high-performance
database applications, so-called federated databases are distributed across many
servers \citep{schwartz2012}) as a central database server, and the others
connect to it over the network. The server must be configured to allow remote
connections \citep{mysql2013}.

\begin{figure}[h]
\centering
\def\svgwidth{0.7\textwidth}
\input{img/server-client.pdf_tex}
\caption[MySQL server-client model]{MySQL server-client model. Multiple
computers (clients) can connect to a single database server and run independent
analyses on the same data.}
\label{fig:server-client}
\end{figure}


When using a multi-client setup with standard desktop computers like described
above, it is beneficial to designate one computer exclusively for the MySQL
server instance, and not run \pname analyses on that machine. The large amount
of data that is sent to and from the server, and the overhead due to table
(re-)indexing take up the majority of the CPU capacity. In order to not hinder
the other computer's analysis performance, the server computer should be able to
dedicate its entire resources to managing the MySQL database (\ie, no \pname
analysis should be running on the server itself). MySQL itself is multithreaded
and scales well over multiple processors. It can handle hundreds of simultaneous
connections per second without problems. The InnoDB storage engine benefits from
large amounts of RAM because it can cache portions of a table in RAM and thus
does not have to access the hard drive frequently \citep{schneider2005}.
