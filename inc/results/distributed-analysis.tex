\label{sec:distributed-analysis}
The server-client model of MySQL enables a distributed analysis on a computer
network. This is advantageous on either an HPC cluster designed for massively
parallelized computing, where many physical computers are connected in a
low-latency network and communicate with each other, or a network of standard
desktop computers, where the clients communicate with the server only. In either
environment, one computer acts (or multiple computers act: in high-performance
database applications, so-called federated databases are distributed across many
servers \citep{schwartz2012}) as a central database server, and the others
connect to it over the network. The server must be configured to allow remote
connections \citep{mysql2013}.

\begin{figure}[h]
\centering
\def\svgwidth{0.7\textwidth}
\input{img/server-client.pdf_tex}
\caption[MySQL server-client model]{MySQL server-client model. Multiple
computers (clients) can connect to a single database server and run independent
analyses on the same data.}
\label{fig:server-client}
\end{figure}


I applied and tested \pname in a setup as depicted in
\autoref{fig:server-client}. It worked well for the client computers. On the
MySQL server machine, however, \pname analyses were up to five times slower than
on the other client machines. This was due to the MySQL server instance taking
up most of the CPU capacity for managing the database. In section
\autoref{sec:distributed-analysis-discussion}, advantages and considerations of
a networked setup are discussed, as well as future perspectives in integration
of \pname in an HPC environment.

