\label{sec:algorithm}
The \pname package consists of three separate tools:

\begin{description}
	\item[orthograph-manager] is a helper application intended for management of
		the database. It is intended to create the initial database structure, upload
		ortholog set and nucleic or amino acid sequence data and manage present
		ortholog sets.
	\item[orthograph-analyzer] is the actual pipeline that performs the
		bidirectional searches using \tool{hmmsearch} and BLAST. It loads the
		results into the database, but does not attempt to infer ortholog
		relationships. 
	\item[orthograph-reporter] is the reporting program that fetches the data that
		\tool{orthograph-analyzer} stored in the database and establishes ortholog
		relationships by means of an iterative algorithm (see
		\autoref{sec:algorithm-reporting})
\end{description}

The decision to implement three different tools  has advantages over a
monolithic design, where a single application would do every task: database
management is separated from the actual analysis. This distinction is of
particular importance when running \pname in a multi-user environment where the
researcher does not have administrative access to the computer or the database
server. Here, the system administrator can manage the database with
\tool{orthograph-manager} and provide the researcher with the required
environment for analysis. No knowledge of Perl or SQL is required.

The separation of searching and reporting algorithms into discrete programs is
beneficial because it facilitates implementation of different analysis
strategies. The data in the database can be evaluated in multiple ways without
having to run the searches again. \tool{orthograph-reporter} provides the algorithm for
predicting gene orthology that is described in
\autoref{sec:algorithm-reporting}, but for a different question, the data may be
assessed under a different algorithm or with different criteria. This versatile
implementation provides programmers and researchers alike with a flexible
framework for analyses on multiple levels. 

The searching and reporting programs can be chained into a single run with a
simple batch script. The configuration file-driven design facilitates this: a
single configuration file contains the necessary information for all \pname
programs.

In the following subsections, the algorithms for analysis and evaluation are
outlined in detail.
