\label{sec:mysql}
I used the MySQL relational database management system (DBMS) for fast and
efficient sequence data storage and retrieval. It is published under the GPL
license for open source applications. MySQL was installed from the Fedora
repositories using the package manager yum.

The use of a DBMS has a number of advantages over file-based sequence data
storage:

\begin{description}
	\item[Memory efficiency.] The amino acid or nucleotide sequence data do not
		need to be loaded into RAM for fast access since the DBMS manages their
		storage and retrieval.  This becomes especially important when analyzing data
		files larger than the computer's physical memory.
	\item[No redundancy.] If the database is well-designed, every amino acid or
		nucleotide sequence and each orthology relationship is stored in the database
		exactly once with a unique identifier. This also contributes to a better
		performance because the DBMS has to search a smaller search space to find a
		given sequence.
	\item[Performance.] The DBMS is highly optimized for speed and efficiency. If
		configured correctly, complex queries comprising millions of rows are
		executed quickly.
	\item[Flexibility.] SQL queries allow fine-grained, custom-tailored filtering
		and output.
\end{description}

MySQL supports the InnoDB \citep{mysql2013} and MyISAM \citep{mysql2013} storage
engines. InnoDB is a so-called transactional database storage engine and is
fully ACID compliant. ACID stands for Atomicity, Consistency, Isolation and
Durability \citep{haerder1983}. It is a set of database design principles that
emphasize aspects of reliability that are important for data-critical
applications \citep{schwartz2012}:

\begin{description}
	\item[Atomicity.] A transaction must function as a single indivisible unit of
		work. It either can be applied or must be rolled back. There is no
		``partially completed'' transaction.
	\item[Consistency.] The database should always be in a consistent state. Since
		transactions either do or do not succeed (see atomicity), corruption due to
		partial or conflicting alterations is impossible.
	\item[Isolation.] Results of a transaction are invisible to other
		transactions until the transaction is complete. This is true for
		transactions on the same isolation level.
	\item[Durability.] Once commited, changes made by a transaction are permanent.
		This means that the changes must be recorded such that data cannot be lost
		in a system crash. 
\end{description}

The above properties ensure reliability and data integrity. The storage engine
InnoDB also implements row-level locking instead of table-level locking. This
means that when inserting, updating, or deleting rows, not the entire table is
locked, but only the row that is being worked on, enabling parallel transactions
for higher performance. 

MyISAM used to be the default storage engine for MySQL up to version 5.1
\citep{schwartz2012}. It does not support transactions or row-level locks and is
not crash-safe. In terms of performance, most of the time InnoDB is the better
choice. However, due to its much less complex nature, MyISAM tables can be
compressed to take up much less space both on disk and in memory. In addition,
MyISAM allows the disabling of indices during data upload, which---especially
with large datasets such as nucleotide or amino acid sequences from a genome
project and in tables with many indices---can be a performance gain
\citep{mysql2013}. 

I made use of both the InnoDB and MyISAM storage engines on different
tables, depending on their specific role and use.
