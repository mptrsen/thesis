In this project, a MySQL relational database management system (RDBMS) is used
for fast and efficient sequence data storage and retrieval. The introduction of
a RDBMS has a number of advantages over file-based sequence storage:

\begin{description}
	\item[Memory efficiency.] The sequence data does not need to be loaded into
		RAM for fast access since the DBMS manages sequence storage and retrieval.
		This becomes especially important when analyzing data files larger than the
		computer's physical memory.
	\item[No redundancy.] If the database is well-designed, every sequence and
		each orthology relationship is stored in the database exactly once with a
		unique identifier. This also contributes to a good performance because the
		DBMS has to search a smaller space to find a given sequence.
	\item[Performance.] The DBMS is highly optimized for speed and efficiency. If
		configured correctly, complex joins comprising millions of rows are executed
		quickly due to the use of B-tree indices (\cite{comer1979}).
	\item[Flexibility.] SQL queries allow fine-grained, custom-tailored filtering
		and output.
\end{description}

