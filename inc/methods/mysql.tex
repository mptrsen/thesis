In this project, a MySQL relational database management system (RDBMS) is used
for fast and efficient sequence data storage and retrieval. It is published
under the GPL license for open source applications. MySQL was installed from the
Fedora repositories using the package manager yum.

The introduction of a RDBMS has a number of advantages over file-based sequence
storage:

\begin{description}
	\item[Memory efficiency.] The sequence data does not need to be loaded into
		RAM for fast access since the DBMS manages sequence storage and retrieval.
		This becomes especially important when analyzing data files larger than the
		computer's physical memory.
	\item[No redundancy.] If the database is well-designed, every sequence and
		each orthology relationship is stored in the database exactly once with a
		unique identifier. This also contributes to a good performance because the
		DBMS has to search a smaller space to find a given sequence.
	\item[Performance.] The DBMS is highly optimized for speed and efficiency. If
		configured correctly, complex joins comprising millions of rows are executed
		quickly due to the use of B-tree indices (\cite{comer1979}).
	\item[Flexibility.] SQL queries allow fine-grained, custom-tailored filtering
		and output.
\end{description}

This version of MySQL supports the InnoDB and MyISAM storage engines. InnoDB is
a so-called transactional database storage engine and is ACID compliant. ACID
stands for Atomicity, Consistency, Isolation and Durability \citep{haerder1983}.
This is a set of database design principles that emphasize aspects of
reliability that are important for data-critical applications
\citep{schwartz2012}:

\begin{description}
	\item[Atomicity.] A transaction must function as a single indivisible unit of
		work. It either can be applied or must be rolled back. There is no
		``partially completed'' transaction.
	\item[Consistency.] The database should always be in a consistent state. Since
		transactions either do or do not succeed due to atomicity, corruption due to
		partial or conflicting alterations is impossible.
	\item[Isolation.] Results of a transaction are invisible to other
		transactions until the transaction is complete. This is true for
		transactions on the same isolation level.
	\item[Durability.] Once commited, changes made by a transaction are permanent.
		This means that the changes must be recorded such that data cannot be lost
		in a system crash. 
\end{description}

These properties ensure reliability and data integrity. InnoDB also implements
row-level locking instead of table-level locking. This means that when
inserting, updating or deleting rows, not the entire table is locked, but only
the row that is being worked on, enabling highly parallel transactions. 

MyISAM used to be the default storage engine for MySQL up to version 5.1
\citep{schwartz2012}. It does not support transactions or row-level locks and is
not crash-safe. In terms of performance, most of the time InnoDB is the better
choice. However, due to its much less complex nature, MyISAM tables can be
compressed to take up much less space both on disk and in memory. In addition,
MyISAM allows the disabling of indices during data upload, which, especially
with large datasets such as entire transcriptomes (as will be explained later),
can be a performance gain. 

\pname uses both InnoDB and MyISAM storage engines on different tables. 
