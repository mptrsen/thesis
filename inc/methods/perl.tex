The program \pname, that I developed and that I present here and all related
helper scripts were written in Perl. When using existing Perl modules written
by others, I only relied on core modules, \ie, modules that are present in any
standard Perl distribution (see table \ref{tab:modules}), with the exception of
\code{IO::Tee} \citep{shan2001}, which was downloaded from the Comprehensive
Perl Archive Network (CPAN\footnote{\url{http://search.cpan.org}}) and bundled
with \pname. All other modules were custom-written by me.

\begin{table}
	\caption[Perl modules used in this project]{Perl modules used this project.
		Modules above the line are present in any standard Perl distribution, with the
		exception of \code{IO::Tee}, which is bundled with \pname. Modules listed
		below the line were written by me.
	}
	\centering
	\begin{tabularx}{\textwidth}{p{0.28\textwidth} p{0.68\textwidth}}
		\hline
		Module & Description \\
		\hline
		\code{autodie}        & Automatically calls \code{die} on I/O errors \\
		\code{strict}         & Enforces safe programming conventions \\
		\code{warnings}       & Enables warning messages during runtime \\
		\code{Archive::Tar}   & Handles tar archive files \\
		\code{Carp}           & Extended warning and error output, including call stack \\
		\code{Config}         & Allows reading of the system configuration \\
		\code{DBD::mysql}     & MySQL database driver \\
		\code{DBI}            & Database interface \\
		\code{Digest::SHA}    & Implements the SHA hashing algorithm \\
		\code{File::Basename} & Provides a function that returns the basename of a file\\
		\code{File::Path}     & Create or remove directory trees \\
		\code{File::Spec}     & Platform-independent path handling; loaded by \code{File::Path} \\
		\code{File::Temp}     & Handles temporary files and directories \\
		\code{FindBin}        & Provides the location of the script during compile time \\
		\code{IO::Dir}        & Object-oriented access to directories \\
		\code{IO::File}       & Object-oriented access to files \\
		\code{IO::Tee}        & Enables multiplexed output to multiple buffers \\
		\code{Time::HiRes}    & Enables high-resolution timer \\
		\hline
		\code{Seqload::Fasta}        & Object-oriented access to fasta files \\
		\code{Wrapper::Hmmsearch}    & Object-oriented wrapper interface to HMMsearch \\
		\code{Wrapper::Blastp}       & Object-oriented wrapper interface to blastp \\
		\code{Wrapper::Mysql}        & Wrapper functions for MySQL \\
		\code{Orthograph::Functions} & Functions for all \pname tools \\
		\code{Orthograph::Config}    & Reads configuration files and provides configuration variables \\
	\end{tabularx}
	\label{tab:modules}
\end{table}


Well aware of the fact that with BioPerl\footnote{\url{http://www.bioperl.org}}
there exists an extensive library of Perl modules for bioinformatic tasks, I
chose to write my own wrapper modules. BioPerl provides an exhaustive,
object-oriented interface to biological sequence handling and wrappers for
various programs, including HMMER3, Exonerate and BLAST. However, I wanted
\pname to remain as independent as possible from the development of other
software packages. In addition, BioPerl's multiple layers of object-oriented
abstraction carry a lot of overhead, which I wanted to avoid if possible. This
ensures a minimum of overhead while providing the required features and maximum
control.
