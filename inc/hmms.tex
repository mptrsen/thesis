\section{Hidden Markov models}

To avoid statistical bias by uneven phylogenetic representation when creating a
hidden Markov model (HMM), \texttt{hmmbuild} from the HMMer package
(\cite{Eddy2009}) uses a position-based weighting scheme (\cite{Henikoff1994})
by default: The sequence distances are not calculated based on the sequences as
a whole, but use a diversity measure for each position in the alignment. From
the paper:

\begin{quote}
	A simple method to represent the diversity at a position is to award each
	different residue an equal share of the weight, and then to divide that
	weight equally among the sequences sharing the same residue. So, if in a
	position of a multiple alignment, $r$ different residues are represented, a
	residue represented in only one sequence contributes a score of $l/r$ to that
	sequence, whereas a residue represented in $s$ sequences contributes a score
	$l/rs$ to each of the $s$ sequences. For each sequence, the contributions
	from each position are summed to give a sequence weight.
\end{quote}

Thus, if two sequences are very similar in a particular domain, the Henikoff
weighting scheme would penalize that by weighting them down position-wise.
Highly diverse positions, on the other hand, would receive a bonus. 

In comparison to two species $a$ and $b$ from the same family, the remotely
related species $a$ and $c$ have more divergent sequences. Highly conserved
domains may still be similar, but for the most part, more divergence is
expected. Under the Henikoff weighting scheme, closely related domains will
receive a penalty, while divergent ones are upweighted. Because $a$ and $b$ are
expected to have more positions in common, which results in downweighting of a
larger percentage of their entire sequences, these two sequences are each
``worth'' less than the more remotely related sequence $c$. Two identical
sequences would each receive half the weight that one sequence would.

\section*{Other weighting schemes}

\texttt{hmmbuild} can also use other weighting schemes, namely a BLOSUM matrix
and a tree-based weighting scheme used by \cite{gerstein1992}. Neither the
BLOSUM matrix nor the tree-based weighting scheme use position-based sequence
weights. From the HMMer user guide about the BLOSUM approach:

\begin{quote}
	Use the same clustering scheme that was used to weight data in calculating
	BLOSUM subsitution matrices (\cite{henikoff1994}). Sequences are
	single-linkage clustered at an identity threshold (default 0.62; see
	\lstinline{--wid}) and within each cluster of $c$ sequences, each sequence
	gets relative weight $1/c$. 
\end{quote}

The Gerstein tree-based weighting scheme uses a distance tree based on percent
identity. It upweights sequences that are close to the root of the tree and
diverse from the rest of the alignment. The tree is constructed by arithmetic
averaging of pairwise distances, where the distances are measured using percent
residue identity (see \cite{Nei1987} and \cite{Sneath1973}). Nodes in the tree
are visited subsequently, going from the ones closest to 100 \% identity (leaf
groups) to 0 \% identity (root). For each node, relative weight is added to both
left and right subtrees according to the length of the edge connecting them to
the node. The length is measured from the previously visited node in the tree.
At each node, sequence weights are updated accordingly:

\begin{equation}
	w(s,b) \leftarrow w(s,b) + D(b)F(s,b)
	\label{eq:gerstein}
\end{equation}

$b$ is L or R for left or right subtree, $s$ runs over all sequences in a
subtree, $D(b)$ is the edge length to be apportioned, $w(s,b)$ is the current
weight of the sequence in subtree $t$, $F(s,b)$ is the weight fraction of the
sequence $s$ in the subtree $t$ according to formula \ref{eq:gerstein2} in the
appendix.

