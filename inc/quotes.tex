\thispagestyle{empty}
\null
\vfill
%--------------------------------------------------
% \begin{quote}
% 	\emph{Mutation:} it is the key to our evolution. It has enabled us to evolve
% 	from a single-celled organism into the dominant species on the planet. This
% 	process is slow, and normally taking thousands and thousands of years. But
% 	every few hundred millennia, evolution leaps forward.\\
% 	\null\hfill--- \citet{singer2000}
% \end{quote}
%-------------------------------------------------- 
\begin{quote}
The Haggunenons of Vicissitus Three have the most impatient chromosomes of any
life form in the Galaxy. Whereas most races are content to evolve slowly and
carefully over thousands of generations, discarding a prehensile toe here,
nervously hazarding another nostril there, the Haggunenons would do for Charles
Darwin what a squadron of Arcturan stunt apples would have done for Sir Isaac
Newton. Their genetic structure, based on the quadruple sterated octohelix, is
so chronically unstable, that far from passing their basic shape onto their
children, they quite frequently evolve several times over lunch. But they do
this with such reckless abandon that if, sitting at table, they are unable to
reach a coffee spoon, they are liable without a moment's consideration to mutate
into something with far longer arms \dots but which is probably quite incapable
of drinking coffee.\\
\null\hfill--- The Hitchhiker's Guide to the Galaxy \citep{adams1979}
\end{quote}
