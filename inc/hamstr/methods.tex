\section{Methods}

\subsection{Installation}

The \hamstr package v1.8 by \cite{Ebersberger2009} was downloaded from \url{http://www.deep-phylogeny.org/hamstr} and unpacked. Since it is a collection of Perl scripts, no compilation was necessary.

The wise2 package v2.2.0 by \cite{Birney2004} was downloaded from \url{http://www.ebi.ac.uk/Tools/Wise2/}, compiled and installed. To fix a conflicting definition of \lstinline{getline()} in \file{src/HMMer2/sqio.c} (line 232) of the wise2 package, the function was renamed to \lstinline{genewise_getline()}. 

hmmer v3.0 by \cite{Eddy2009} was downloaded from \url{http://hmmer.janelia.org/}, compiled and installed.

Additionally, the \file{insecta\_hmmer3-2} core ortholog set was downloaded and installed.

\subsection{Testing}

The included test set was run according to the instructions in the \file{README} file that came with the \hamstr package. The command was as follows:

\begin{verbatim}
$ ../bin/hamstrsearch_local-hmmer3.v8.pl -est -sequence_file=testset.fa \
	-taxon=test -hmmset=modelorganisms_hmmer3 -refspec=DROME -representative \
	-hmm=317.hmm
\end{verbatim}

The options and flags have the following meanings:
\begin{description}
	\item[-est] Conduct an EST, not a protein analysis.
	\item[-sequence\_file=] The file containing the EST data.
	\item[-taxon=] String containing a user-selected taxon name. This is used in the FASTA file headers.
	\item[-hmmset=] The HMM set to use. \hamstr looks for HMM sets in the directory \file{core\_orthologs}.
	\item[-refspec=] Reference species among the core orthologs to use. There has to exist a directory under \file{core\_orthologs/} with that name.
	\item[-representative] If \hamstr detects more than one hit, the program will choose the best of the three hits as representative. If two or more hits match to non-overlapping parts of the reference protein, these hits will be kept and subsequently concatenated. The FASTA header of the concatenated hit sequence will then contain information which sequences have been concatenated, and how long they are.
	\item[-hmm=] Use one specified HMM instead of a set of HMMs. There has to exist a file with that name under \file{core\_orthologs/} and it has to end in \file{.hmm}.
\end{description}
For a complete listing of available options see the \hamstr \file{README} file in the appendix. The test run completed without errors.

\subsection{Real-Life Tests}

Real-life files were tested: An EST file containing sequences of varying length
(103-21135 bp) was searched using the following command.

\begin{verbatim}
../bin/hamstrsearch_local-hmmer3.v8_mpmod.pl \
-sequence_file=cleaned/Mengenilla_12_5_mio_PE_velvet_contigs.fa -est \
-taxon=Mengenilla_trica -hmmset=insecta_hmmer3-2 -refspec=trica_1449 \
-representative -strict -eval=1e-05 -hmmset=insecta_hmmer3-2 
\end{verbatim}

The test run completed without problems.

\subsection{Benchmarks}

The \hamstr pipeline uses standard UNIX tools for certain tasks, \eg filtering
content from a file using \code{grep} or line editing with \code{sed}. A
benchmark was conducted to compare the time efficiency of \code{sed} to the
native regular expression parser in Perl: A file containing one line was read
into memory, the line was changed back and forth, and the file was written back
to disk. This was repeated $10^4$ times. The process was timed using the
\code{time} tool.

Additionally, the \code{grep} tool was benchmarked. A Perl script was used that
searches a file for an expression (listing \ref{lst:grep-perl}), and the GNU
Regular Expression Parser \code{grep} (listing \ref{lst:grep-grep}). The test
was done $10^4$ times and timed using the \code{time} tool.

\subsubsection{Using UNIX tools}

\lstinputlisting[language=bash, caption=Substitution using UNIX tools, label=lst:subst-sed]{scripts/subst.sh}

The \code{sed} variant yielded the following result:

\begin{verbatim}
$ bash subst.sh  16% cpu 18.596 total
$ bash subst.sh  16% cpu 18.515 total
$ bash subst.sh  16% cpu 18.502 total
-------------------------------------
Average:                 18.538 total
\end{verbatim}

\lstinputlisting[caption=Searching a file using UNIX tools, label=lst:grep-grep]{scripts/grep.sh}

\begin{verbatim}
./grep.sh Perl ../latex/inc/methods.tex  17% cpu 15.829 total
./grep.sh Perl ../latex/inc/methods.tex  17% cpu 15.288 total
./grep.sh Perl ../latex/inc/methods.tex  17% cpu 15.552 total
-------------------------------------------------------------
Average:                                         15.556 total
\end{verbatim}

\subsubsection{Using native Perl functions}

\lstinputlisting[caption=Substitution using native Perl functions, label=lst:subst-perl]{scripts/subst.pl}

The variant using native Perl functions yielded the following result:

\begin{verbatim}
./subst.pl  97% cpu 0.524 total
./subst.pl  98% cpu 0.447 total
./subst.pl  97% cpu 0.483 total
-------------------------------
Average:            0.485 total
\end{verbatim}

\lstinputlisting[caption=Searching a file using native Perl functions, label=lst:grep-perl]{scripts/grep.pl}

\begin{verbatim}
./grep.pl Perl ../latex/inc/methods.tex  99% cpu 4.397 total
./grep.pl Perl ../latex/inc/methods.tex  99% cpu 4.478 total
./grep.pl Perl ../latex/inc/methods.tex  99% cpu 4.607 total
------------------------------------------------------------
Average:                                         4.494 total
\end{verbatim}

The native Perl solutions (listings \ref{lst:subst-perl}, \ref{lst:grep-perl})
use much more processor time, but are also much faster than the solutions using
UNIX tools (listings \ref{lst:subst-sed}, \ref{lst:grep-grep}). It can be
assumed that a Perl script using UNIX tools will run slower than a Perl script
that employs its built-in functions for the same task. For this reason
\pname will use built-in Perl functions wherever possible.

\subsection{\code{genewise} benchmarks}

\code{Genewise} (\cite{Birney2004}) was tested for reliability with randomly
altered sequences. A script (listing \ref{lst:randomframeshifts} on page
\pageref{lst:randomframeshifts}) was used to insert frameshift errors simulated
by random nucleotides, at a random interval and with fixed probability, into
the sequence. 

\subsubsection{\code{genewise} vs. \code{exonerate}}

Genewise can be replaced using Exonerate (\cite{Slater2005}), which runs about
ten times faster and allows the output of the frameshift-corrected,
corresponding nucleotide sequences (cDNA) along with the peptide sequences.
A test using a peptide and a nucleotide sequence yielded the following result:

\begin{verbatim}
exonerate --exhaustive --model protein2genome:bestfit --verbose 0 
--showalignment no apime_411985.fa anva_L12193T1.fa --ryo 
"Score: %s\n
>%qi_%ti_%tcb-%tce_cdna\n%tcs//\n
>%qi[%qab:%qae]_query\n%qas//\n
>%ti[%tab:%tae]_target\n%tas" 
\end{verbatim}

In addition, Exonerate comes with a set of fasta file manipulation tools.
Fastatranslate translates a nucleotide fasta file in all six possible reading
frames. Because it is written in C++, this implementation is much faster than the
one used in \hamstr, which is written in Perl. Fastatranslate changes the
header from:

\begin{verbatim}
>Original_header
\end{verbatim}

to (\eg, for the third reading frame on the reverse complement strand):

\begin{verbatim}
>Original_header [revcomp]:[translate(3)]
\end{verbatim}

HMMer3's hmmsearch treats everything after the first whitespace in a header as a
sequence description field, not as part of a unique header. This can lead to
non-unique sequence identifiers in the hmmsearch results. 

\label{uniq}
This problem becomes irrelevant if all sequences are assigned a unique ID
without whitespace before using them in the analysis. For later reference, the
original ID can be written back after the analysis. 
