\label{sec:distributed-analysis-discussion}
When using a multi-client setup with standard desktop computers like described
in \autoref{sec:distributed-analysis}, it is beneficial to designate one
computer exclusively for the MySQL server instance, and not run additional
\pname analyses on that machine. The large amount of data that is sent to and
from the server, and the overhead due to table indexing take up the majority of
the CPU capacity. In order to not hinder the other computer's analysis
performance, the server computer should be able to dedicate its entire resources
to managing the MySQL database (\ie, no \pname analysis should be running on the
server itself). The MySQL server itself is multithreaded and scales well over
multiple processors. It can handle hundreds of simultaneous connections per
second without problems. The InnoDB storage engine benefits from large amounts
of RAM because it can cache portions of a table in RAM and thus does not have to
access the hard drive frequently \citep{schneider2005}.

When planning an integration of \pname in a HPC environment, additional factors
have to be considered. HPC clusters are optimized to provide a high computing
capacity by relying on a massively parallelized architecture. MySQL database
servers, especially when running large InnoDB engine tables exclusively, benefit
the most from a large memory pool. However, memory is valuable for all
applications that run on HPC clusters, and allocating a large portion of the
available memory to the database server is not feasible. Thus, the database
server must be externalized (see \autoref{fig:cluster} for a possible
configuration). This requires a low-latency network connection so that the
database connection does not become a performance bottleneck.

\begin{figure}[h]
	\centering
	\def\svgwidth{0.8\textwidth}
	\input{img/cluster.pdf_tex}
	\caption[Possible integration of a database in a grid computing environment]{
		Possible integration of a database server in a grid computing environment. In
		a computing grid, the individual compute nodes communicate with each other
		and exchange data. They are coordinated by the control node, which is the only
		interface to both the user and the database.
	}
	\label{fig:cluster}
\end{figure}

