The algorithm in \pname has been distributed into three programs:
\tool{orthograph-manager}, \tool{orthograph-analyzer}, and
\tool{orthograph-reporter}. The decision to implement three different tools
has advantages over a monolithic design, where a single application would do
every task: database management is separated from the actual analysis. This
distinction is of particular importance when running \pname in a multi-user
environment where the researcher does not have administrative access to the
computer or the database server. Here, the system administrator can manage the
database with \tool{orthograph-manager} and provide the researcher with the
required environment for analysis. No knowledge of Perl or SQL is required.

The separation of searching and reporting algorithms into discrete programs is
beneficial because it facilitates implementation of different analysis
strategies. The data in the database can be evaluated in multiple ways without
having to run the searches again. \tool{orthograph-reporter} provides the
algorithm for predicting gene orthology that is described in
\autoref{sec:algorithm-reporting}, but for a different question, the data may be
assessed under a different algorithm or with different criteria. This versatile
implementation provides programmers and researchers alike with a flexible
framework for analyses on multiple levels. 

The searching and reporting programs can be chained into a single run with a
simple batch script. The configuration file-driven design facilitates this: a
single configuration file contains the necessary information for all \pname
programs.

