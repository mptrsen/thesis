\label{sec:future-devel}
In its present state, \pname has been tested to work as described. However, the
triangulation approach that was treaded by \pname's archetype \hamstr, offers
numerous opportunities to future extension. The modular design of \pname
facilitates integration of different algorithms as well as modification and
exchange of existing modules in order to implement improved methods.

Performance considerations have always been a driving force during development.
At the time of this writing, a single run of \pname takes about as long as a
single analysis with \hamstr on standard desktop computer hardware. The time
required for a \pname analysis depends on the size of the ortholog set: the more
ortholog groups, the more HMMs are used to search the transcriptome sequence
library, and the more ortholog candidates have to be evaluated. Additionally,
the database size is relevant when deleting data from previous analyses, and
when uploading new transcriptome data (see subsection
\ref{sec:mysql-performance}). 
