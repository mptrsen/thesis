In \hamstr, one BLAST database has to be maintained for each reference taxon.
This leads to the problem of BLAST scores and e-values not being normalized:
since the BLAST databases for the reference proteomes that are used in \hamstr
are of differing sizes, the e-values, which depend on the database size, cannot
be compared across reference taxa. To generate a ``normalized'' score and
e-value, \hamstr takes an additional comparison step using a pairwise alignment
with ClustalW.

In contrast to \hamstr, \pname uses one BLAST database comprising all reference
taxa proteomes. E-values are comparable across taxa since the database is of
constant size. Additionally, only one BLAST search is required instead of $n$
searches, where $n$ is the number of reference taxa. This leads to a potential
performance boost of up to $n-1$ compared to \hamstr.
