\label{sec:mysql-performance-discussion}
During testing, MySQL performance became a bottleneck because of the re-indexing
issues of MyISAM, which take increasingly long with growing table size
(see subsection \autoref{sec:mysql-performance}. A solution to this issue is to
partition the tables ``ests'', ``hmmsearch'', and ``blast'' into individual
tables for each query dataset. Transcriptome sequence data as generated within
the 1KITE project have a number of sequences in the order of $10^5$ to
$5\times10^5$, which amounts to the same number of records in the database. That
means, in the worst case, the database in its current form and without
administrative access to server variables can only store up to about ten query
datasets before it becomes slow.

Besides the indexing strategy and the query design, the database structure plays
a major role in terms of performance. Especially the tables ``hmmsearch'' and
``blast'' become excessively large after a few analyses: with an ortholog set of
4,000 OGs, if each of the 4,000 HMM searches obtains an average of 20 hits
during the HMM search, and each of the 20 BLAST searches obtains an average of
50 hits, this amounts to $4,000 \times 20 = 8 \times 10^4$ rows in the table
``hmmsearch'' and $4 \times 10^6$ rows in the table ``blast''.  The rows
themselves do not contain much data (only 131 B on average), but due to the
large number of records, InnoDB performance with the current schema does not
scale well. Insertion of new data does not become noticably slow; however,
deleting old analysis results takes very long. The InnoDB cluster index
physically orders the table based on the primary key or the first unique key it
can utilize. When one row is removed, the entire table is reordered on the hard
drive for speed and defragmentation. With increasing table size, this operation
takes exponentially long (\autoref{fig:delete-time}).

The simplest and most performant solution to this problem is to create
individual tables for each query taxon. Deleting and recreating a table is much
faster than deleting a large portion out of a huge table, especially when there
are indices involved. 

\begin{figure}[h]
	\centering
	\def\svgwidth{0.4\textwidth}
	\input{img/delete-time.pdf_tex}
	\caption[Deletion time on large InnoDB tables]{
		InnoDB performance on deleting rows in large tables with many indices. With
		increasing table size, the transaction takes longer because the InnoDB
		cluster index has to physically reorder the table contents on the hard
		drive.
	}
	\label{fig:delete-time}
\end{figure}

