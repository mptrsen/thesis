To avoid statistical bias by uneven phylogenetic representation when creating a
HMM, a position-based weighting scheme \citep{henikoff1994} is used by default:
The sequence distances are not calculated based on the sequences as a whole, but
use a diversity measure for each position in the alignment from which the HMM is
being trained. From the paper by \citet{henikoff1994}:

\begin{quote}
	A simple method to represent the diversity at a position is to award each
	different residue an equal share of the weight, and then to divide that weight
	equally among the sequences sharing the same residue. So, if in a position of
	a multiple alignment, $r$ different residues are represented, a residue
	represented in only one sequence contributes a score of $l/r$ to that
	sequence, whereas a residue represented in $s$ sequences contributes a score
	$l/rs$ to each of the $s$ sequences. For each sequence, the contributions from
	each position are summed to give a sequence weight.
\end{quote}

Thus, if two sequences are very similar in a particular domain, the Henikoff
weighting scheme penalizes by weighting them down position-wise. Highly diverse
positions, on the other hand, receive a bonus. 

In comparison to two species $a$ and $b$ from the same family, the remotely
related species $a$ and $c$ have more divergent sequences. Highly conserved
domains may still be similar, but for the most part, more divergence is
expected. Under the Henikoff weighting scheme, closely related domains will
receive a penalty, while divergent ones are upweighted. Because $a$ and $b$ are
expected to have more positions in common, which results in downweighting of a
larger percentage of their entire sequences, these two sequences are each
``worth'' less than the more remotely related sequence $c$. Under this weighting
scheme, two identical sequences would each receive half the weight that one
sequence would.


