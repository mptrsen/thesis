The \pname algorithm is different from the one that is implemented in \hamstr:
It does not assign transcript sequences to ortholog groups (OG) based on a
per-OG basis, but instead first collects all possible graph edges  and traverses
them by ascending \tool{hmmsearch} e-value (see \autoref{fig:orthograph-graph}).
This has two advantages: firstly, it assures that the transcript sequences are
mapped to the most relevant OG, and not the one that was processed first.
Secondly, by removing the transcript sequences from the list of candidate
transcripts, redundant assignment is precluded, \ie, the same transcript cannot
be mapped to multiple OGs or vice versa.
