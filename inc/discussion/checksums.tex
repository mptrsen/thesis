SHA-256 digests are generated using a cryptographic hashing algorithm described
by \citet{gallagher2008}. For a consistent and reproducible state of the
nucleotide or amino acid sequence data during all analysis steps, it must be
guaranteed that no two digests, \ie, two sequence identifiers, are ever
identical. The SHA-256 hashing algorithm generates a digest that is 160 bits, or
40 hexadecimal characters in length. The probability $p$ of a hash collision
(\ie, two hashed elements returning the same digest) in $n$ elements is

\begin{equation}
p \ge \frac{n (n-1)}{2} \times \frac{1}{2^b}
\label{eq:hashcollision}
\end{equation}

where $b$ is the number of bits generated by the hash function. There need to be
more than $1.7 \times 10^{15}$ objects for the SHA-256 hashes to exceed a collision
probability of $10^{-18}$. Since the hash space is expected to contain a number
of objects in the range of $10^6$ to $10^{12}$, it is statistically safe to
assume that every digest is unique. 

